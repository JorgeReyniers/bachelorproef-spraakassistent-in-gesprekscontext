%==============================================================================
% Sjabloon onderzoeksvoorstel bachelorproef
%==============================================================================
% Gebaseerd op LaTeX-sjabloon ‘Stylish Article’ (zie voorstel.cls)
% Auteur: Jens Buysse, Bert Van Vreckem

\documentclass[fleqn,10pt]{voorstel}
%\addbibresource{voorstel}
\bibliography{voorstel.bib}


%------------------------------------------------------------------------------
% Metadata over het voorstel
%------------------------------------------------------------------------------

\JournalInfo{HoGent Bedrijf en Organisatie}
\Archive{Bachelorproef 2018 - 2019} % Of: Onderzoekstechnieken

%---------- Titel & auteur ----------------------------------------------------

% TODO: geef werktitel van je eigen voorstel op
\PaperTitle{Stemgestuurde technologie als ondersteuning van de begeleider in de jeugdzorg}
\PaperType{Onderzoeksvoorstel Bachelorproef} % Type document

% TODO: vul je eigen naam in als auteur, geef ook je emailadres mee!
\Authors{Jorgé Reyniers\textsuperscript{1}} % Authors
\CoPromotor{Nog niet aangewezen}
\affiliation{\textbf{Contact:}
  \textsuperscript{1} \href{mailto:jorge.reyniers.w5872@student.hogent.be}{jorge.reyniers.w5872@student.hogent.be};
}

%---------- Abstract ----------------------------------------------------------

\Abstract{    
Afgelopen decennia heeft technologie gezorgd voor heel wat positieve veranderingen in het dagdagelijkse leven van de mens. De mogelijkheden blijven eindeloos tot op de dag van vandaag. Kinderen in de jeugdzorg hebben een individuele begeleider nodig. Helaas kan die begeleider niet continu aanwezig zijn in het centrum waar het kind verblijft. In dit onderzoek wordt er gezocht naar een vervanger voor de afwezige begeleider in de vorm van een Smart Speaker, een apparaat met stemgestuurde technologie. Er zal getoetst worden bij de betrokkenen naar de vereisten van de applicatie. Tegelijk zal er onderzocht worden welke mogelijkheden van spraakgestuurde technologie er bestaan en welke wordt gekozen om de applicatie te ontwikkelen. Na deze analyse zal de ontwikkeling van de applicatie beginnen. Het doel is een eerste versie van een applicatie te ontwikkelen die ingezet kan worden in het werkveld. De verwachtingen liggen hoog. De bedoeling is om na het einde van het onderzoek de ontwikkeling van de applicatie niet te beëindigen. De verantwoordelijkheid kan doorgegeven worden naar een instantie die zich inzet voor innovatie in de zorgsector.
}

%---------- Onderzoeksdomein en sleutelwoorden --------------------------------
% TODO: Sleutelwoorden:
%
% Het eerste sleutelwoord beschrijft het onderzoeksdomein. Je kan kiezen uit
% deze lijst:
%
% - Mobiele applicatieontwikkeling
% - Webapplicatieontwikkeling
% - Applicatieontwikkeling (andere)
% - Systeembeheer
% - Netwerkbeheer
% - Mainframe
% - E-business
% - Databanken en big data
% - Machineleertechnieken en kunstmatige intelligentie
% - Andere (specifieer)
%
% De andere sleutelwoorden zijn vrij te kiezen

\Keywords{Applicatieontwikkeling. Jeugdzorg --- Voice Assistant --- Individuele begeleiding} % Keywords
\newcommand{\keywordname}{Sleutelwoorden} % Defines the keywords heading name

%---------- Titel, inhoud -----------------------------------------------------

\begin{document}

\flushbottom % Makes all text pages the same height
\maketitle % Print the title and abstract box
\tableofcontents % Print the contents section
\thispagestyle{empty} % Removes page numbering from the first page

%------------------------------------------------------------------------------
% Hoofdtekst
%------------------------------------------------------------------------------

% De hoofdtekst van het voorstel zit in een apart bestand, zodat het makkelijk
% kan opgenomen worden in de bijlagen van de bachelorproef zelf.
%---------- Inleiding ---------------------------------------------------------

\section{Introductie} % The \section*{} command stops section numbering
\label{sec:introductie}
Kinderen in jeugdcentra hebben een individuele begeleider nodig. De begeleider is verantwoordelijk voor het kind de aandacht, begeleiding en zorg op maat te geven die hij of zij nodig heeft. De begeleider is de vertrouwensfiguur voor het kind.
Een probleem dat zich voordoet is dat de begeleider alleen aanwezig is tijdens zijn werkuren. De vertrouwenspersoon kan voor een periode wegvallen en tijdens deze periode is het kind zijn steunfiguur kwijt.

Spraakgestuurde technologie is aan een opmars bezig. In de Verenigde Staten hebben één op de vijf volwassenen toegang tot een smart speaker met stemassistent.~\autocite{Passies2018} In België is het gebruik van een slimme luidspreker nog niet van de grond. Dit kan sinds de komst van de Nederlandse versie van Google Assistent wel eens gaan veranderen. Ook Google's slimme luidspreker, Google Home, kwam eind oktober voor het eerst op de Nederlandse markt.~\autocite{Haenen2018} Het apparaat bevat de nieuwe Google Assistent en is daardoor de eerste smart speaker die Nederlands begrijpt. Consumenten hebben een heel andere band met een voice assistant dan met een ander apparaat. Ze spreken over hun apparaat alsof het een mens is.\autocite{Schueler2018} Een gesprek voeren met een apparaat kan zorgen voor een persoonlijke band, iets wat bij de bekende smartphone ontbreekt.

%Iets over dat ik gemotiveerd ben omdat het gloednieuw is dus heel veel mogelijkheden ontstaan enzoo allemaal nog nieuw he.
Het commerciële gebruik van deze technologie in ons land is dus gloednieuw. Het onderzoek over wat we met deze technologie allemaal kunnen verwezenlijken, kan voor veel mogelijkheden zorgen.

%Hier beginnen over de doelstelling van het onderzoek en onderzoeksvragen
Doel van het toegepast onderzoek is om na te gaan of stemgestuurde technologie gebruikt kan worden om kinderen in jeugdcentra te ondersteunen. Het is niet de bedoeling om een apparaat te ontwikkelen die de begeleider in de toekomst zal vervangen. Het kan het kind helpen ondersteunen wanneer zijn individuele begeleider niet aanwezig is in het jeugdcentrum.
Het onderzoek bevat volgende onderzoeksvraag: Hoe kan stemgestuurde technologie helpen met de begeleiding van kinderen in jeugdcentra bij afwezigheid van de individuele begeleider? 
De onderzoeksvraag bestaat uit twee deelvragen, elk binnen zijn aparte domein. Op orthopedagogisch vlak luidt de vraag: Welke noden zijn er om een stemgestuurde applicatie te ontwikkelen? Vertrekkende vanuit deelvraag 1 gaat het binnen het technologische domein over: Welke spraaktechnologie biedt de meeste mogelijkheid voor een goede oplossing?

%---------- Stand van zaken ---------------------------------------------------

\section{State-of-the-art}
\label{sec:state-of-the-art}

De laatste jaren wordt er meer en meer onderzoek gedaan naar hoe IT en zorg kunnen samenwerken. Zo kwam de overheid met e-health, een elektronisch platform waar alle betrokkenen in de volksgezondheid gegevens kunnen uitwisselen.

Een belangrijk begrip waar onderzoek naar gedaan wordt is Blended Care. Geestelijke gezondheidszorg, ondersteund door IT. Het beste proberen gebruiken van beide werelden. De patiënt krijgt online een behandeling, maar wordt daarnaast ook nog steeds ondersteund door een begeleider. De mix van online en face-to-face therapie heeft al bewezen veel voordelen te hebben. Uit het SROI-verslag van \textcite{Stil2016} concludeert men dat gemiddeld over vijf jaar, de investeringen voor Blended Care 2,2 keer dit bedrag aan maatschappelijke baten opleveren. De cliënt krijgt de mogelijkheid om zelf aan zijn geestelijke gezondheid te werken tussen sessies met de begeleider door. Wat er voor zorgt dat de cliënt vertrouwen krijgt in het zelfstandig omgaan met zijn gezondheid.~\autocite{Wentzel2016} Uit het net vermelde onderzoek blijkt ook dat er wel nog meer onderzoek nodig is om te bepalen welke precieze mix de voorkeur krijgt van cliënt en begeleider in bepaalde situaties.

Eén van de grootste behaalde projecten die technologie in de zorgsector toepast is Zora, de zorgrobot. Een slimme robot die nu wordt ingezet in verschillende zorginstellingen. Het wordt beklemtoond dat het niet de bedoeling is om mensen in de zorg te vervangen, maar te begeleiden.~\autocite{Grypdonck2015} Er verschijnen veel positieve berichten over Zora en haar functionaliteiten, maar er is wel zeker één groot hekelpunt, de kostprijs. Die ligt namelijk rond de vijftienduizend euro.~\autocite{Jongejan2016}

Hoewel het dus bewezen is dat zorg en technologie samengaan, is er nog geen specifiek onderzoek gedaan naar het gebruik van een smart speaker in de jeugdzorg. Dit nicheproduct kan mogelijks toegevoegde waarde bieden aan het grotere geheel. Zo is ook al bewezen dat IT ervoor zorgt dat de drempel om de stap naar hulp te zetten lager wordt door de anonimiteit die ermee gepaard gaat.~\autocite{Stil2016} Het gebruik van een smart speaker biedt veel mogelijkheden om de zorgsector te verbeteren.

Een andere vraag is of mensen klaar zijn om gebruik te maken van deze technologie. De jeugd staat meer open voor het gebruik van nieuwe technologische middelen, maar ook de begeleiders moeten hier mee akkoord gaan. De heer Buysse, mijn promotor voor deze bachelorproef, is aan een project bezig over faciliterende IT bij individuele begeleidingsgesprekken in de jeugdzorg. Uit navraag bij ex-cliënten en begeleiders bleek dat IT geen oplossing is om het gesprek te vervangen. Vandaar ook dit specifieke onderzoek naar stemgestuurde technologie als ondersteuning bij afwezigheid van de individuele begeleider.

% Voor literatuurverwijzingen zijn er twee belangrijke commando's:
% \autocite{KEY} => (Auteur, jaartal) Gebruik dit als de naam van de auteur
%   geen onderdeel is van de zin.
% \textcite{KEY} => Auteur (jaartal)  Gebruik dit als de auteursnaam wel een
%   functie heeft in de zin (bv. ``Uit onderzoek door Doll & Hill (1954) bleek
%   ...'')

%---------- Methodologie ------------------------------------------------------
\section{Methodologie}
\label{sec:methodologie}
Eerst en vooral zal er parallel onderzoek gedaan worden naar een antwoord op de twee deelvragen van het onderzoek.
Op orthopedagogisch domein wordt er aan veldonderzoek gedaan. Er wordt zoveel mogelijk gekeken naar wat de vraag is bij begeleider en kind. Wat zijn de vereisten voor de applicatie? Dit kan gedaan worden door kwalitatief onderzoek in de vorm van interviews en/of enquêtes. Die kunnen afgenomen worden bij alle betrokkenen. Dit kan ver gaan, maar er zal vooral gefocust worden op de directe doelgroep, de begeleiders en kinderen in jeugdcentra. Er kunnen ook gegevens verzameld worden door te observeren.
%Literatuurstudie over het onderwerp? Gaat er wel wat te vinden zijn?
%andere betrokkenen: ortho's

Op technologisch domein wordt er kennis verworven over de verschillende mogelijkheden van spraakgestuurde technologie. Een vergelijkende studie die de voor- en nadelen van de bestaande technologiën afweegt, zal beslissen welke technologie er gekozen wordt. Met de gekozen optie zal er uiteindelijk verder gewerkt worden.
%En literatuurstudie vermelden?

Nadat beide evaluaties worden gematcht, zal er een proof of concept opgesteld worden die zal beslissen of de applicatieontwikkeling wel of niet wordt gestart. Als de applicatie mag ontwikkeld worden zal er dikwijls naar feedback van de directe betrokkenen gepolst worden. De bedoeling is om dan tegen het einde van het onderzoek een eerste versie van de applicatie te ontwikkelen waarop later kan worden voortgebouwd.
%Wat als er beslist wordt dat er beter geen applicatie wordt gemaakt?

%---------- Verwachte resultaten ----------------------------------------------
\section{Verwachte resultaten}
\label{sec:verwachte_resultaten}

Concrete resultaten zijn moeilijk te voorspellen aangezien er geen metingen en simulaties worden gedaan in het onderzoek. Uit interviews en observaties kunnen er onvoorspelbare en uiteenlopende resultaten ontstaan waar de onderzoeker zelf nooit zou kunnen opkomen.

Er kan wel al nagedacht worden over welke noden de betrokkenen kunnen hebben. Zo kan er nood zijn aan een smart speaker die een gesprek kan aangaan met het kind wanneer hij/zij er naar vraagt. Indien gewenst kan het gesprek opgenomen worden zodat dit later kan beluisterd worden door de begeleider. Het kind geeft beter eerst de toestemming om het gesprek op te nemen zodat het apparaat het vertrouwen van het kind niet schaadt.
Het kan een optie zijn om het apparaat te doen reageren op crisismomenten van een kind. Inspelen op het moment dat het kind een crisismoment beleeft, kan een belangrijke verantwoordelijkheid van het apparaat worden.
Het kan goed zijn dat het apparaat gegevens uit het verleden kan ophalen om het kind zo goed mogelijk te ondersteunen.
De technologie die zal gebruikt worden lijkt vooral te neigen naar de Google Home omdat dit voorlopig de eerste Nederlandstalige Smart Speaker is op de markt.

%---------- Verwachte conclusies ----------------------------------------------
\section{Verwachte conclusies}
\label{sec:verwachte_conclusies}

Er wordt verwacht dat uit de interviews, enquêtes en observaties ideeën voortvloeien die de functionaliteiten van de applicatie zullen bepalen. Het zal vast en zeker een uitdaging worden om het apparaat als een vertrouwenspersoon te doen fungeren voor het kind.
Kritiek kan een hindernis worden tijdens het onderzoek.
Orthopedagogen kunnen ervan overtuigd zijn dat een vertrouwensband alleen kan ontstaan tussen mensen. Dergelijke personen kunnen weigerachtig staan tegenover het gebruik van technologie in hun domein.
Dat er tegenstanders zijn bewijst ook het bestaan van de website www.zorgictzorgen.nl.
Toch wordt er verwacht dat er een goede samenwerking tussen de IT'er en de orthopedagoog zal ontstaan en er een eerste productversie zal ontwikkeld worden tegen het einde van het onderzoek.
Een verwachting in de toekomst, na de bachelorproef, is dat een instantie de ontwikkeling van de applicatie verder in handen neemt. Er kan bijvoorbeeld contact opgenomen worden met het zorglab van Vives. Het zorglab verdiept en verbreedt de expertise over zorgtechnologie en interprofessioneel samenwerken en vertaalt deze kennis via kennisvalorisatie naar eindgebruikers, zorgverleners, bedrijven en het onderwijs.~\autocite{Vives}




%------------------------------------------------------------------------------
% Referentielijst
%------------------------------------------------------------------------------
% TODO: de gerefereerde werken moeten in BibTeX-bestand ``voorstel.bib''
% voorkomen. Gebruik JabRef om je bibliografie bij te houden en vergeet niet
% om compatibiliteit met Biber/BibLaTeX aan te zetten (File > Switch to
% BibLaTeX mode)

\phantomsection
\printbibliography[heading=bibintoc]

\end{document}
