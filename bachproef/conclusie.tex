%%=============================================================================
%% Conclusie
%%=============================================================================

\chapter{Conclusie}
\label{ch:conclusie}

Uit het onderzoek is gebleken dat de algemene spraakkwaliteit van de Engelstalige Alexa en Google Assistant hoger liggen dan die van de Nederlandstalige Google Assistant. Bij elke eigenschap scoorden ze significant hoger dan de Nederlandstalige assistent, met uitzondering van de verstaanbaarheid van Alexa. Voor deze eigenschap kan niet bewezen worden dat de score bij Alexa significant hoger ligt dan bij Google Assistant NL.
Omdat de applicatie moet helpen met eerste hulp verlenen in België is het absoluut nodig dat de assistent Nederlands kan. Google Assistant en Siri zijn de enige twee spraakassistenten die op dit moment een Nederlandse versie hebben. Omdat een ontwikkelaar Siri enkel kan integreren in een bestaande mobiele applicatie gaat de voorkeur eerder naar Google Assistant. Daarnaast is er ook nog het nieuws dat er binnenkort een Vlaamse versie komt van de Google Assistant. Google doet dus uitschijnen dat ze met hun Google Home als eerste willen proberen intrekken in de Vlaamse woonkamers. Deze factoren hebben meegedragen aan het antwoord dat de Google Assistant het meest geschikt is om te helpen bij het verlenen van eerste hulp bij ongevallen. We kunnen besluiten dat er uiteindelijk voor de keuze van de meest geschikte spraakassistent geen rekening is gehouden met de resultaten van het vergelijkend onderzoek.

Dit geldt echter alleen voor België en voor nu. Er kan ook op wereldwijd niveau gekeken worden naar het ontwikkelen van een spraakgestuurde eerstehulp-applicatie. In dit geval kan er wel rekening worden gehouden met de resultaten van het vergelijkend onderzoek voor de landen waar vooral Engels wordt gesproken. De twee Engelstalige assistenten verschillen echter maar aanzienlijk op één eigenschap, namelijk het tempo. De participanten vonden dat Google Assistant op een gepaster tempo praat dan Alexa. Een niet gepast tempo was een tempo dat ervaren werd als te snel of te traag. Als de meest geschikte spraakassistent alleen o.b.v. deze resultaten wordt gekozen, dan zou Google Assistant opnieuw als winnaar uit de bus komen. De kans bestaat ook dat Alexa of andere spraakassistenten later met een Nederlandse of Vlaamse versie op de markt komen. Het zou interessant zijn om dan dezelfde vergelijking te maken tussen Nederlandstalige assistenten. 

Een correct onderzoek naar de kwaliteit van spraakherkenning verloopt moeilijker dan gedacht. Er is besloten om het aantal fouten die elke assistent heeft gemaakt bij het omvormen van spraak naar tekst niet te tonen of te vergelijken. Waarom dit deel van het onderzoek moeilijk uit te voeren was op een correcte manier, staat beschreven in \ref{ss:een moeilijk te voeren onderzoek}. Er kunnen lessen uit geleerd worden voor toekomstige onderzoekers die gaan werken in hetzelfde onderzoeksveld.

Na het schrijven van deze bachelorproef is er het plan om de eerste stappen te zetten in het ontwikkelen van de applicatie. De meest geschikte spraakassistent staat vast en de structuur is getekend. De eerste versie kan ingezet worden om feedback te verzamelen van testgebruikers. Daarnaast kan het Rode Kruis gecontacteerd worden om te polsen wat ze daar van de applicatie vinden. Mogelijks zijn ze geïnteresseerd in een verdere samenwerking voor het ontwikkelen van de applicatie.

Dat de spraaktechnologie in een stroomversnelling is beland is duidelijk. In Amerika hebben 1 op 3 inwoners al aankopen met hun smart speaker. Alexa van Amazon heeft zichzelf daar ontpopt als favoriet door onder andere de koppeling aan de . Een sterkte die ze hier nog niet kunnen inzetten door het gebrek aan een Nederlandse versie. Daardoor ziet Google zijn kans en doet ze haar best om hier te groeien als de nummer één in spraakassistentie met Google Assistant en de slimme Smart Speaker Google Home. Spraakbesturing heeft veel potentieel, zeker in combinatie met the internet of things. Zonder zelfs te moeten opstaan kun je de rolluiken openen, een film starten op netflix of de lampen dimpen. Het klinkt aanlokkelijk. Zal de Smart Speaker eerst een hebbeding worden voor de early adapters en vooral, zal er met enkele jaren sprake zijn van een derde digitale golf, na het internet en de smartphone?



 

