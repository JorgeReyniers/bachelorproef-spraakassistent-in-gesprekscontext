%%=============================================================================
%% Methodologie
%%=============================================================================

\chapter{Methodologie}
\label{ch:methodologie}

%% TODO: Hoe ben je te werk gegaan? Verdeel je onderzoek in grote fasen, en
%% licht in elke fase toe welke stappen je gevolgd hebt. Verantwoord waarom je
%% op deze manier te werk gegaan bent. Je moet kunnen aantonen dat je de best
%% mogelijke manier toegepast hebt om een antwoord te vinden op de
%% onderzoeksvraag.

\section{Vergelijking van stemgestuurde technologieën}
\label{sec:vergelijking van stemgestuurde technologieën}

\subsection{Algemeen}
\label{sec:algemeen}
Welke talen kan het nu en in de nabije toekomst ondersteunen.

\subsection{Gebruik van de Assistant}
\label{sec:gebruik van de assistant}
\textbf{gemiddelde response time}
Vraag 1
Resultaat: Boxplot
Vraag 2
Resultaat: Boxplot
Vraag 3
Resultaat: Boxplot

\textbf{Welke middelen zijn er nodig om het te gebruiken?}
Resultaat: Beschrijving van de nodige middelen

\textbf{Hoe uitgebreid is de functionaliteit?}
Resultaat: score op 20 (op hoeveel vragen van de 20 kan hij een correct antwoord geven)

\textbf{Gemiddeld succesratio van het correct begrijpen van een vraag}
Correct gestelde vraag
vb. Wie is de eerste president van Amerika?
Resultaat: Succesratio ?\%

Vraag met een grammaticale fout
vb. Wie is president het eerste Amerika?
Resultaat: Succesratio ?\%

Vraag in het dialect
vb. Wien es den iersten president van Amerika?
Resultaat: Succesratio?\%

\subsection{Ontwikkelen van een applicatie}
\label{sec:ontwikkelen van een applicatie}
Welke middelen kan je gebruiken om het te ontwikkelen?
Resultaat: Tekst over de te gebruiken middelen

Hoe uitgebreid is de documentatie?
Resultaat: Tekst over de documentatie

Welke programmeertalen worden aangeboden?
Resultaat: Opsomming programmeertalen

\subsection{Vergelijking}
\label{sec:vergelijking tussen de onderzochte bedrijven}
\textbf{Response Time}
Resultaat: Grafiek

\textbf{Middelen}
Resultaat: Tabel

\textbf{Functionaliteit}
Resultaat: Grafiek

\textbf{Begrijpen}
Resultaat: Grafiek

\textbf{Ontwikkelen van een eigen applicatie}
Resultaat: Tabel

\section{Orthopedagogisch onderzoek}
---Nog te bepalen---

\lipsum[21-25]

