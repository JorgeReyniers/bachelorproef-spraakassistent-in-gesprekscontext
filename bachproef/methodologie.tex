%%=============================================================================
%% Methodologie
%%=============================================================================

\chapter{Methodologie}
\label{ch:methodologie}

\section{Vergelijking van stemgestuurde technologieën}
\label{sec:vergelijking van stemgestuurde technologieën}

\subsection{Algemeen}
\label{sec:algemeen}

Waarom voor deze technologiëen gekozen om te onderzoeken. Waarom geen Siri, Mycroft en Microsft Cortana.

\begin{center}
    \begin{tabular}{ | l | l | l | l |}
        \hline
        Assistant & Smart speaker & Prijs & Nederlandse ondersteuning\\ \hline
        Google Assistant & Google Home (Mini) & +-60 EUR en +-150 EUR & Ja \\ \hline
        Amazon Alexa & Amazon Echo (dot) & +-74 EUR en +-122 EUR & Neen \\ \hline
        Mycroft & Mark I & 132,55 EUR & Neen \\ \hline
    \end{tabular}
\end{center}

\subsection{Gebruik van de Assistant}
\label{sec:gebruik van de assistant}
functionaliteiten die belangrijk zijn:
automatisch een telefoontje plegen: Google Assistant gaat niet

Wat is de response time? Boxplot

Welke middelen zijn er nodig om het te gebruiken?

Hoe uitgebreid is de functionaliteit?

Hoe vaak wordt hij begrepen bij een lange vraag? (op hetzelfde smartphone device) succesratio

Hoe verstaanbaar vindt een gebruiker het antwoord? Score op correctheid en natuurlijkheid

Tot welke afstand blijft de assistant je begrijpen? (op hetzelfde smartphone device)

\subsection{Ontwikkelen van een applicatie}
\label{sec:ontwikkelen van een applicatie}
Welke middelen kan je gebruiken om het te ontwikkelen?

Hoe uitgebreid is de documentatie?

Welke programmeertalen worden aangeboden?

Kan de applicatie automatisch een nummer bellen?

\subsection{Vergelijking van de verschillende technologieën}
\label{sec:vergelijking van de onderzochte technologieën}
\textbf{Response Time}
Resultaat: Grafiek

\textbf{Middelen}
Resultaat: Tabel

\textbf{Functionaliteit}
Resultaat: Grafiek

\textbf{Begrijpen}
Resultaat: Grafiek

\textbf{Ontwikkelen van een eigen applicatie}
Resultaat: Tabel

\section{Orthopedagogisch onderzoek}
\label{Orthopedagogisch onderzoek}
\subsection{De zoektocht naar een gepaste doelgroep}
\label{De zoektocht naar een gepaste doelgroep}

\subsubsection{Ondersteuning van de begeleider in de jeugdzorg}
\label{ondersteuning van de begeleider in de jeugdzorg}
De aanleiding van dit onderzoek was een onderzoek van -citeer onderzoek mr. Buysse- over faciliterende IT bij individuele begeleidingsgesprekken in de jeugdzorg. Uit het onderzoek ontstond er twijfel over het gebruik van een spraakassistent als ondersteuning bij het individuele begeleidingsgesprek tussen de persoonlijke begeleider en het kind. Door de twijfel werd er dan ook beslist om niet verder te gaan met spraaktechnologie, maar met andere digitale tools.

Om er toch zeker van te zijn dat er geen mogelijkheden waren, ontstond deze bachelorproef. Echter, na een eerste gesprek met Iris Storme, docent orthopedagogie binnen Hogeschool Gent en tevens mede-researcher van -citeer onderzoek mr. Buysse- werden voor mij de beweegredenen voor het afkeuren van spraaktechnologie binnen hun onderzoek snel duidelijk. 

Als je denkt aan mensen met een visuele of fysieke beperking komen er snel mogelijkheden naar boven. Denk maar aan het controleren van apparaten met een eenvoudig stemcommando. Deze personen kunnen technologie als een mogelijke oplossing zien, waardoor zij, en de begeleiders, dit gemakkelijker kunnen omarmen.
Daartegenover staat de bijzondere jeugdzorg, waar de spraakassistent eerder ondersteuning zou bieden in de emotionele problematiek en de jongeren net hun façade nodig hebben om overeind te blijven. Deze doelgroep stelt zich niet zo graag kwetsbaar op en ervaart het praten over gevoelens eerder als een drempel. De bijzondere jeugdzorg lijkt op het eerste zicht een minder relevante doelgroep.

Dit werd allemaal vastgesteld door het intuïtieve gevoel van de ondervraagde. Dit was voor mij persoonlijk voldoende om na te gaan denken over een nieuwe doelgroep.

\subsubsection{Ondersteuning van personen met het syndroom van Down}
\label{ondersteuning van personen met het syndroom van Down}
Ik wijzigde mijn doelgroep naar personen die geboren zijn met trisomie 21, ook wel het syndroom van Down genoemd. Uit een eerste opzoeking stelde ik de volgende mogelijkheden.

Personen met het Downsyndroom worden geboren met een verstandelijke beperking. Er kan gekeken worden naar welke noden uit die verstandelijke beperking vloeien, bijvoorbeeld moeite met rekenen, en hoe een spraakassistent hier ondersteuning kan bieden. Dit kan ook veel verder gaan als in vb. het helpen met zelfstandig wonen.

Daarnaast zijn er ook mogelijke bijkomende aandoeningen zoals een minder goed geheugen, coeliakie, slaapapneu, oogafwijkingen of een gedragsstoornis. Hier kan spraakassistentie mogelijks ook ondersteuning in bieden. Ik denk aan bijvoorbeeld interactieve activiteiten voor het stimuleren van de motoriek, het geheugen of het spraakvermogen, helpen herinneren aan belangrijke taken, helpen herinneren aan wat ze wel of niet mogen eten, stimuleren van een vast slaappatroon, enzovoort.

--Bronnen vermelden--

Dit waren nog maar losse ideeën die ontstonden uit een eerste verkenning over personen met het gendefect. Het werd duidelijk dat hier zeker mogelijkheden waren, dus was de volgende stap om hierin gaan te verdiepen door interviews af te nemen van mensen die een persoonlijke ervaring hebben met deze doelgroep.

Echter, mijn eerste aanvraag voor een interview aan iemand van wie haar dochter geboren is met trisomie 21 stootte direct op weerstand. De respons die ik kreeg ging over het gevaar van het veralgemenen. Het is een complexe materie omdat het gaat over een combinatie van samenkomende symptomen en kenmerken. De effecten van het gendefect kunnen in verschillende gradaties voorkomen, waardoor elke persoon die het gendefect bezit uniek moet bekeken worden.