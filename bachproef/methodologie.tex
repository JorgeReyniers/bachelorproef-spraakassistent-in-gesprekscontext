%%=============================================================================
%% Methodologie
%%=============================================================================

\chapter{Methodologie}
\label{ch:methodologie}

\section{Vergelijking van stemgestuurde assistenten}
\label{sec:vergelijking van stemgestuurde assistenten}

\section{Het verloop van het onderzoek}
De doelgroep van het onderzoek zijn Vlamingen die Nederlands spreken en enige kennis hebben van de Engelse taal. Daarnaast zijn er geen andere vereisten.

De onderzoeker volgt voor elke afname een vast stappenplan dat bestaat uit twee delen.

In deel één wordt elke deelnemer verteld dat hij drie vragen zal stellen aan een assistent, terwijl deze tegelijkertijd worden opgenomen. De deelnemer kan elke vraag aflezen van een blad. In werkelijkheid worden de vragen wel opgenomen, maar spreekt de participant niet tegen een assistent. Dit wordt geveinsd om ervoor te zorgen dat de manier waarop de deelnemer de vragen stelt gelijkaardig is aan de manier waarop hij ze zou stellen aan een assistent. De deelnemer krijgt nooit een antwoord te horen, waardoor de mogelijkheid ontstaat dat hij steeds meer zijn best doet de vraag duidelijker uit te spreken.

De opgenomen audiofragmenten worden later gebruikt om de assistenten hun spraakherkenningsniveau te meten. Elke assistent krijgt afzonderlijk de kans om naar de audiofragmenten te luisteren en ze te vertalen naar tekst. Door het aantal fouten in de vertaalde teksten te tellen kan dan gemeten worden over welk niveau van spraakherkenning de assistenten beschikken. Elke assistent hoort de gelijke vraag van de deelnemer in een redelijk identieke omgeving, wat niet het geval zou zijn bij het werken met rechtstreekse vragen van de deelnemers aan de assistent. 

Deel twee is het onderzoek naar het spraakniveau van de assistenten. Hiervoor stelt de deelnemer dezelfde drie vragen van in deel één aan drie spraakassistenten. Hij wordt aangespoord om goed te luisteren naar de antwoorden van de assistent. De participant leest eerst de vragenlijst die hij na het luisteren zal beantwoorden. De volgorde is in elke afname van het onderzoek gelijk. Eerst wordt er geluisterd naar en beoordeeld over de spraakkwaliteit van Alexa. Daarna is de Engelse en Nederlandse Google Assistant aan de beurt. De deelnemer krijgt tijdens elk beoordeling de kans vorige beoordelingen te herzien.

Details over het onderzoek inclusief stappenplan, vragenlijsten en gebruikte commando's zijn te vinden in bijlage.

\section{De keuze van de spraakassistenten}
Omdat het gaat over een vergelijkend onderzoek, is het belangrijk in weinig veranderlijke omstandigheden te werken. Om te vermijden dat de hardware een invloed heeft op de resultaten worden de assistenten gebruikt op hetzelfde toestel, de Xiami Redmi Note 4. Hierdoor kregen bepaalde assistenten geen kans meer om opgenomen te worden in het onderzoek. Bixby is enkel beschikbaar op apparaten van Samsung zoals Siri enkel beschikbaar is voor Apple-gebruikers. Daarnaast heeft het ontwikkelen van eigen applicaties beperkingen voor beide assistenten. VERTEL HIER NOG OVER WELKE BEPERKINGEN PRECIES


\subsection{Algemeen}
\label{sec:algemeen}
Een eerste stap in mijn onderzoek was het selecteren van de te onderzoeken assistenten. De kandidaten waren de meest prestigieuze assistenten van dit moment. Uiteindelijk heb ik besloten Amazon Alexa te vergelijken met de Engelse en Nederlandse versie van de Google Assistant.

Het onderzoek vergelijkt algemeen twee eigenschappen van de assistenten: de Speech-To-Text en de Text-To-Speech. 

Waarom voor deze technologieën gekozen om te onderzoeken. Waarom geen Siri, Mycroft en Microsoft Cortana en Bixby.

\begin{center}
    \begin{tabular}{ | l | l | l | l |}
        \hline
        Assistant & Smart speaker & Prijs & Nederlandse ondersteuning\\ \hline
        Google Assistant & Google Home (Mini) & +-60 EUR en +-150 EUR & Ja \\ \hline
        Amazon Alexa & Amazon Echo (dot) & +-74 EUR en +-122 EUR & Neen \\ \hline
        Samsung Bixby & Samsung Galaxy Home & Nog niet op de markt & Neen\\ \hline
    \end{tabular}
\end{center}

\subsection{Gebruik van de assistant}
\label{sec:gebruik van de assistant}

\subsubsection{Individueel onderzoek}
functionaliteiten die belangrijk zijn:
automatisch een telefoontje plegen: Google Assistant gaat niet

Wat is de response time? Boxplot

Op welke devices is de assistant beschikbaar?

Tot welke afstand blijft de assistant je begrijpen? (op hetzelfde smartphone device)

Onderzoek naar algoritmes van de assistants, werken de verschillende talen met hetzelfde algoritme

\subsubsection{Vergelijkend onderzoek van de assistenten bij derden}
-Hier komt uitleg tegen proefpersoon wat er zal gebeuren tijdens het onderzoek, wat een spraakassistent is, ..
Drie goede vragen zoeken: één die moeilijk uit te spreken is, één met een lang antwoord

EERSTE DEEL
In te vullen door de proefpersoon:
Voor het onderzoek:
Leeftijd, geslacht, Niveau Engels, Ervaring met Personal Assistants

Na het onderzoek:
Vond je het ongemakkelijk om tegen een apparaat te praten?

Hoe goed heeft de assistent de deelnemer begrepen?
Hoe goed heeft de deelnemer de assistent begrepen?/hoe natuurlijk klinkt de stem?

Score op 5 op de natuurlijkheid van het antwoord. Klinkt de stem menselijk, heeft het antwoord een logische zinsbouw, legt de assistent de juiste klemtonen, laat hij de juiste pauzes, praat hij op een goed tempo.

In te vullen door mezelf:
Was er ruis aanwezig bij commando aan GA, Alexa of Cortana?
Hoeveel commando's heeft de assistant begrepen?
Ging de gebruiker voorover buigen?

TWEEDE DEEL
Assistent een tekst laten voorlezen en de proefpersoon laten luisteren.

\subsection{Ontwikkelen van een applicatie}
\label{sec:ontwikkelen van een applicatie}
Welke middelen kan je gebruiken om het te ontwikkelen?

Hoe uitgebreid is de documentatie?

Welke programmeertalen worden aangeboden?

Kan de applicatie automatisch een nummer bellen?

\subsection{Vergelijking van de verschillende technologieën}
\label{sec:vergelijking van de onderzochte technologieën}
\textbf{Response Time}
Resultaat: Grafiek

\textbf{Middelen}
Resultaat: Tabel

\textbf{Functionaliteit}
Resultaat: Grafiek

\textbf{Begrijpen}
Resultaat: Grafiek

\textbf{Ontwikkelen van een eigen applicatie}
Resultaat: Tabel

\section{Orthopedagogisch onderzoek}
\label{Orthopedagogisch onderzoek}
\subsection{De zoektocht naar een gepaste doelgroep}
\label{De zoektocht naar een gepaste doelgroep}

\subsubsection{Ondersteuning van de begeleider in de jeugdzorg}
\label{ondersteuning van de begeleider in de jeugdzorg}
De aanleiding van dit onderzoek was een onderzoek van -citeer onderzoek mr. Buysse- over faciliterende IT bij individuele begeleidingsgesprekken in de jeugdzorg. Uit het onderzoek ontstond er twijfel over het gebruik van een spraakassistent als ondersteuning bij het individuele begeleidingsgesprek tussen de persoonlijke begeleider en het kind. Door de twijfel werd er dan ook beslist om niet verder te gaan met spraaktechnologie, maar met andere digitale tools.

Om er toch zeker van te zijn dat er geen mogelijkheden waren, ontstond deze bachelorproef. Echter, na een eerste gesprek met Iris Storme, docent orthopedagogie binnen Hogeschool Gent en tevens mede-researcher van -citeer onderzoek mr. Buysse- werden voor mij de beweegredenen voor het afkeuren van spraaktechnologie binnen hun onderzoek snel duidelijk. 

Als je denkt aan mensen met een visuele of fysieke beperking komen er snel mogelijkheden naar boven. Denk maar aan het controleren van apparaten met een eenvoudig stemcommando. Deze personen kunnen technologie als een mogelijke oplossing zien, waardoor zij, en de begeleiders, dit gemakkelijker kunnen omarmen.
Daartegenover staat de bijzondere jeugdzorg, waar de spraakassistent eerder ondersteuning zou bieden in de emotionele problematiek en de jongeren net hun façade nodig hebben om overeind te blijven. Deze doelgroep stelt zich niet zo graag kwetsbaar op en ervaart het praten over gevoelens eerder als een drempel. De bijzondere jeugdzorg lijkt op het eerste zicht een minder relevante doelgroep.

Dit werd allemaal vastgesteld door het intuïtieve gevoel van de ondervraagde. Dit was voor mij persoonlijk voldoende om na te gaan denken over een nieuwe doelgroep.

\subsubsection{Ondersteuning van personen met het syndroom van Down}
\label{ondersteuning van personen met het syndroom van Down}
Ik wijzigde mijn doelgroep naar personen die geboren zijn met trisomie 21, ook wel het syndroom van Down genoemd. Uit een eerste opzoeking stelde ik de volgende mogelijkheden.

Personen met het Downsyndroom worden geboren met een verstandelijke beperking. Er kan gekeken worden naar welke noden uit die verstandelijke beperking vloeien, bijvoorbeeld moeite met rekenen, en hoe een spraakassistent hier ondersteuning kan bieden. Dit kan ook veel verder gaan als in vb. het helpen met zelfstandig wonen.

Daarnaast zijn er ook mogelijke bijkomende aandoeningen zoals een minder goed geheugen, coeliakie, slaapapneu, oogafwijkingen of een gedragsstoornis. Hier kan spraakassistentie mogelijks ook ondersteuning in bieden. Ik denk aan bijvoorbeeld interactieve activiteiten voor het stimuleren van de motoriek, het geheugen of het spraakvermogen, helpen herinneren aan belangrijke taken, helpen herinneren aan wat ze wel of niet mogen eten, stimuleren van een vast slaappatroon, enzovoort.

--Bronnen vermelden--

Dit waren nog maar losse ideeën die ontstonden uit een eerste verkenning over personen met het gendefect. Het werd duidelijk dat hier zeker mogelijkheden waren, dus was de volgende stap om hierin gaan te verdiepen door interviews af te nemen van mensen die een persoonlijke ervaring hebben met deze doelgroep.

Echter, mijn eerste aanvraag voor een interview aan iemand van wie haar dochter geboren is met trisomie 21 stootte direct op weerstand. De respons die ik kreeg ging over het gevaar van het veralgemenen. Het is een complexe materie omdat het gaat over een combinatie van samenkomende symptomen en kenmerken. De effecten van het gendefect kunnen in verschillende gradaties voorkomen, waardoor elke persoon die het gendefect bezit uniek moet bekeken worden.