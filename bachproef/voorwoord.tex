%%=============================================================================
%% Voorwoord
%%=============================================================================

\chapter*{Woord vooraf}
\label{ch:voorwoord}

Bij de start van mijn bachelorproef kreeg ik al snel te maken met enkele problemen. Ik werd door een gesprek met Iris Storme, docent orthopedagogie aan de Hogeschool Gent, overtuigd om mijn doelgroep zoals die oorspronkelijk stond beschreven in mijn voorstel te veranderen. Na enige twijfel besloot ik om te gaan onderzoeken welke mogelijkheden een spraakassistent kan bieden aan mensen met het Syndroom van Down. Ik werd al snel opnieuw overtuigd een andere doelgroep te gaan zoeken. Na enkele weken was ik er nog steeds niet in geslaagd om een doelgroep vast te leggen en begon ik stilaan kopzorgen te krijgen. Tot ik op de ochtend van 2 april op de trein een artikel las over een mobiele eerstehulp-applicatie die was gebouwd door het Rode Kruis. De geschikte toepassing voor een spraakassistent was gevonden. Bij deze wil ik graag Iris Storme en Margot De Donder bedanken om mij in de juiste richting te sturen.

Ik wou me graag eens volledig in één onderwerp verdiepen. Iets wat de afgelopen semesters niet mogelijk was door de verscheidenheid aan vakken. Doorheen het semester was ik dan ook voortdurend met het onderwerp bezig. Terwijl ik op mijn stage bij de VRT werkte aan een applicatie voor de Google Assistant, gebruikte ik mijn uren ernaast om te werken aan mijn bachelorproef. Op mijn stage heb ik kennis opgedaan over de ontwikkeling van een spraakgestuurde applicatie. Voor algemene vragen over de wereld van spraaktechnologie kon ik steeds terecht bij mijn co-promotor Ronny Pringels. Bij deze wil ik hem daar ook graag voor bedanken.

De promotor, Jens Buysse, heeft ervoor gezorgd dat we meerdere keren met enkele studenten samenkwamen om het verloop van onze bachelorproef te bespreken. Ik wil graag de medestudenten bedanken voor het delen van hun leerrijke ervaringen tijdens hun proces. Meneer Buysse was hier steeds aanwezig om te sturen en vragen te beantwoorden. Naast de bijeenkomsten heeft hij ook geholpen door feedback te geven, te adviseren, de juiste richting te tonen en vragen te beantwoorden. Bij deze zou ik meneer Buysse graag willen bedanken voor de begeleiding gedurende mijn bachelorproef.

Ook bedankt aan de 30 vrijwilligers die tijd namen om een proef af te nemen voor mijn onderzoek. Daarnaast wil ik mijn ouders bedanken om me steeds de goede zorgen te geven zodat ik me volledig kon concentreren op mijn bachelorproef. Ten slotte wil ik mijn vriendin bedanken om me het afgelopen semester steeds te steunen.