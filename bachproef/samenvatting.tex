%%=============================================================================
%% Samenvatting
%%=============================================================================

% TODO: De "abstract" of samenvatting is een kernachtige (~ 1 blz. voor een
% thesis) synthese van het document.
%
% Deze aspecten moeten zeker aan bod komen:
% - Context: waarom is dit werk belangrijk?
% - Nood: waarom moest dit onderzocht worden?
% - Taak: wat heb je precies gedaan?
% - Object: wat staat in dit document geschreven?
% - Resultaat: wat was het resultaat?
% - Conclusie: wat is/zijn de belangrijkste conclusie(s)?
% - Perspectief: blijven er nog vragen open die in de toekomst nog kunnen
%    onderzocht worden? Wat is een mogelijk vervolg voor jouw onderzoek?
%
% LET OP! Een samenvatting is GEEN voorwoord!

Persoonlijke spraakassistenten als Amazon's Alexa, Apple's Siri of de Assistant van Google zijn aan een serieuze opmars bezig in Amerika. Tegenwoordig komen ze ook allemaal met een bijpassende smart speaker waar de spraakassistent is ingebouwd. In België blijft de populariteit nog uit, maar daar kan verandering in komen. Na de aankondiging van Google dat zijn persoonlijke assistent binnenkort een Belgische variant zal krijgen, sijpelen de eerste toepassingen van de grote bedrijven al binnen. Welk doel, naast winst maken, kan een toekomstige applicatie voor de Belgische variant nog hebben? Deze bachelorproef was eerst en vooral een zoektocht naar een doel die voor een bepaalde groep een meerwaarde is en waar op weinig weerstand wordt gebotst. Een bijzonder doelgroep bleek hiervoor niet de beste optie. Daarom is er gekozen om een applicatie te maken die de Vlaming kan helpen bij het verlenen van Eerste Hulp Bij Ongevallen. Veel informatie die nodig is om de applicatie te bouwen is af te leiden uit de recente mobiele applicatie die is gemaakt door het Rode Kruis en hetzelfde doel heeft.

Er is een literatuuronderzoek geschreven over de wereld van spraaktechnologie. Daaruit bleek dat spraakassistenten twee noodzakelijke functies hebben, namelijk spraakherkenning en spraaksynthese. Spraakherkenning of Speech-To-Text is gesproken taal omvormen naar voor de computer leesbare taal. Spraaksynthese is net het omgekeerde en is menselijke spraak gevormd door een computer.

Voor de applicatie kan ontwikkeld worden moet er een assistent gekozen worden waarvoor hij wordt gebouwd. Om er niet zomaar één uit te kiezen is er een vergelijkend onderzoek gevoerd naar de kwaliteit van de spraak en spraakherkenning van drie assistenten.
De spraakkwaliteit van de Engelstalige Google Assistant en Alexa en de Nederlandstalige Google Assistant werden met elkaar vergeleken a.d.h.v. beoordelingen van 30 vrijwilligers die EHBO-vragen hebben gesteld en de antwoorden van de assistenten hebben beluisterd. Voor dit deel van het onderzoek werd een kleine applicatie gemaakt voor elke assistent die als inhoud identiek dezelfde antwoorden gaven. De deelnemers konden een score geven op verstaanbaarheid, levendigheid, menselijkheid, tempo en emotionaliteit. De resultaten zijn barcharts die de drie assistenten vergelijken per eigenschap en boxplots die de vijf eigenschappen vergelijken per assistent. Allemaal op basis van de scores die zijn gegeven door de vrijwilligers.
Uit de resultaten van het eerste deel van het onderzoek worden enkele stellingen bevestigd. Deze tonen aan dat de Nederlandse Google Assistant significant lager scoort op spraakkwaliteit dan de andere twee assistenten. De Engelse assistenten kregen op elke eigenschap telkens een hogere score dan Google Assistant NL. Tussen de Engelse assistenten onderling was er niet veel verschil te merken. Google Assistent scoorde alleen op tempo significant hoger dan Alexa.

Het tweede deel van het onderzoek, het vergelijken van de kwaliteit van de assistenten hun spraakherkenning, was een moeilijkere taak die helaas niet echt in zijn opzet is geslaagd. Ondanks de maatregelen die zijn getroffen om de beïnvloeding van veranderlijke factoren te beperken, is het toch niet gelukt om statistisch correct onderzoek te voeren in het tweede deel. De bedoeling was om het aantal fouten die elke assistent maakt bij het herkennen van de gestelde vraag van de deelnemers te vergelijken. Zo werd er bijvoorbeeld wel aan gedacht om de gestelde vragen allemaal op te nemen, om ervoor te zorgen dat de assistenten later in een geluidsdichte studio identiek dezelfde dertig vragen te horen krijgen uit een speaker. Helaas werd er bijvoorbeeld niet verwacht dat het herkennen van een opgenomen audiofragment aanzienlijk zou verschillen van het herkennen van een rechtstreeks gestelde vraag. Dit is één van de redenen waarom de resultaten niet vergeleken worden. Er wordt wel gekeken naar interessante fouten die de assistenten hebben gemaakt. Uit de fouten die bij dit deel zijn gemaakt kunnen er lessen geleerd worden voor onderzoekers die in de toekomst werken met spraakherkenning.

Uiteindelijk is de keuze ondanks de resultaten toch gevallen op de Nederlandstalige Google Assistant. Dit komt omdat de Nederlandse taal een doorslaggevende factor is voor het verlenen van eerste hulp in België. Daarnaast wijst het erop dat Google als eerste assistant zijn intrek probeert te nemen in de Vlaamse woonkamers. De resultaten blijven bruikbaar bij het kiezen van een spraakassistent voor een eerste hulp-applicatie die wereldwijd wordt aangeboden in de Engelse taal. Als er enkel wordt afgegaan op de resultaten van het onderzoek naar de spraakkwaliteit dan is de Google Assistant de favoriet.

Als laatste is er een kort beeld geschetst van hoe een applicatie voor de Google Assistant werkt en hoe die werking kan verlopen bij de specifieke use case van het onderzoek. Het plan is om in de weken na het schrijven van deze bachelorproef een eerste versie te ontwikkelen van de eerstehulp-applicatie voor de Google Assistant. In de verdere toekomst kan deze versie getest worden. Feedback van gebruikers kan verwerkt worden en bij interesse van het Rode Kruis kan er mogelijks een samenwerking worden aangegaan.

%%---------- Nederlandse samenvatting -----------------------------------------
%
% TODO: Als je je bachelorproef in het Engels schrijft, moet je eerst een
% Nederlandse samenvatting invoegen. Haal daarvoor onderstaande code uit
% commentaar.
% Wie zijn bachelorproef in het Nederlands schrijft, kan dit negeren, de inhoud
% wordt niet in het document ingevoegd.

\IfLanguageName{english}{%
\selectlanguage{dutch}
\chapter*{Samenvatting}
\selectlanguage{english}
}{}

%%---------- Samenvatting -----------------------------------------------------
% De samenvatting in de hoofdtaal van het document



\chapter*{\IfLanguageName{dutch}{Samenvatting}{Abstract}}

