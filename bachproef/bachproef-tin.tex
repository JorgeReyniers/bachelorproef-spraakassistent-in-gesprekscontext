%%=============================================================================
%% LaTeX sjabloon voor bachelorproef, HoGent Bedrijf en Organisatie
%% Opleiding Toegepaste Informatica
%%=============================================================================

\documentclass[fleqn,a4paper,12pt]{book}
\usepackage[backend=biber]{biblatex}
\usepackage{listings}
\lstset{frame=tb,
    language=R,
    breaklines=true,
    showstringspaces=false,
    columns=flexible,
    numbers=none,
    tabsize=3
}
\input{structure}

%%---------- Documenteigenschappen --------------------------------------------
%% TODO: Vul dit aan met je eigen info:

% Je eigen naam
\newcommand{\student}{Jorgé Reyniers}

% De naam van je promotor (lector van de opleiding)
\newcommand{\promotor}{Jens buysse}

% De naam van je co-promotor. Als je promotor ook je opdrachtgever is en je
% dus ook inhoudelijk begeleidt (en enkel dan!), mag je dit leeg laten.
\newcommand{\copromotor}{Ronny Pringels}

% Indien je bachelorproef in opdracht van/in samenwerking met een bedrijf of
% externe organisatie geschreven is, geef je hier de naam. Zoniet laat je dit
% zoals het is.
\newcommand{\instelling}{---}

% De titel van het rapport/bachelorproef
\newcommand{\titel}{Een slimme spraakassistent bij het verlenen van eerste hulp bij ongevallen}

% Datum van indienen (gebruik telkens de deadline, ook al geef je eerder af)
\newcommand{\datum}{31 mei 2019}

% Academiejaar
\newcommand{\academiejaar}{2018-2019}

% Examenperiode
%  - 1e semester = 1e examenperiode => 1
%  - 2e semester = 2e examenperiode => 2
%  - tweede zit  = 3e examenperiode => 3
\newcommand{\examenperiode}{2}

%%=============================================================================
%% Inhoud document
%%=============================================================================

\begin{document}

%---------- Taalselectie ------------------------------------------------------
% Als je je bachelorproef in het Engels schrijft, haal dan onderstaande regel
% uit commentaar. Let op: de tekst op de voorkaft blijft in het Nederlands, en
% dat is ook de bedoeling!

%\selectlanguage{english}

%---------- Titelblad ---------------------------------------------------------
\inserttitlepage

%---------- Samenvatting, voorwoord -------------------------------------------
\usechapterimagefalse
%%=============================================================================
%% Voorwoord
%%=============================================================================

\chapter*{Woord vooraf}
\label{ch:voorwoord}

%% TODO:
%% Het voorwoord is het enige deel van de bachelorproef waar je vanuit je
%% eigen standpunt (``ik-vorm'') mag schrijven. Je kan hier bv. motiveren
%% waarom jij het onderwerp wil bespreken.
%% Vergeet ook niet te bedanken wie je geholpen/gesteund/... heeft

Bij de start van mijn bachelorproef kreeg ik al snel te maken met enkele problemen. Ik werd door een gesprek met Iris Storme, docent orthopedagogie aan de Hogeschool Gent, overtuigd om mijn doelgroep zoals die oorspronkelijk stond beschreven in mijn voorstel te veranderen. Na enige twijfel besloot ik om te gaan onderzoeken welke mogelijkheden een spraakassistent kan bieden aan mensen met het Syndroom van Down. Ik werd al snel opnieuw overtuigd een andere doelgroep te gaan zoeken. Na enkele weken was ik er nog steeds niet in geslaagd om een doelgroep vast te leggen en ik begon stilaan kopzorgen te krijgen. Tot ik op de ochtend van 2 april op de trein een artikel las over een mobiele eerstehulp-applicatie die was gebouwd door het Rode Kruis. De geschikte toepassing voor een spraakassistent was gevonden. Bij deze wil ik graag Iris Storme en Margot De Donder bedanken om mij in de juiste richting te sturen.

Ik wou me graag eens volledig in een onderwerp verdiepen. Iets wat de afgelopen semesters niet mogelijk was door de verscheidenheid aan vakken. Doorheen het semester was ik dan ook voortdurend met het onderwerp bezig. Terwijl ik op mijn stage bij de VRT werkte aan een applicatie voor de Google Assistant, gebruikte ik mijn uren ernaast om te werken aan mijn bachelorproef. Op mijn stage heb ik kennis opgedaan over de ontwikkeling van een spraakgestuurde applicatie. Voor algemene vragen over de wereld van spraaktechnologie kon ik steeds terecht bij mijn co-promotor Ronny Pringels. Bij deze wil ik hem daar ook voor bedanken.

De promotor, Jens Buysse, heeft ervoor gezorgd dat we nu en dan eens met enkele studenten samenkwamen om het verloop van onze bachelorproef te bespreken. Ik wil graag de medestudenten bedanken voor het delen van de leerrijke ervaringen tijdens hun proces. Meneer Buysse was hier steeds aanwezig om te sturen en vragen te beantwoorden. Naast de bijeenkomsten heeft hij ook geholpen door feedback te geven, te adviseren, de juiste richting te tonen en vragen te beantwoorden. Bij deze zou ik meneer Buysse graag willen bedanken voor de begeleiding gedurende mijn bachelorproef.

Ook bedankt aan de 30 vrijwilligers die tijd namen om een proef af te nemen voor mijn onderzoek. Daarnaast wil ik mijn ouders bedanken om me steeds de goede zorgen te geven zodat ik me volledig kon concentreren op mijn bachelorproef. Ten slotte wil ik mijn vriendin bedanken om me het afgelopen semester steeds te steunen.

%%=============================================================================
%% Samenvatting
%%=============================================================================

% TODO: De "abstract" of samenvatting is een kernachtige (~ 1 blz. voor een
% thesis) synthese van het document.
%
% Deze aspecten moeten zeker aan bod komen:
% - Context: waarom is dit werk belangrijk?
% - Nood: waarom moest dit onderzocht worden?
% - Taak: wat heb je precies gedaan?
% - Object: wat staat in dit document geschreven?
% - Resultaat: wat was het resultaat?
% - Conclusie: wat is/zijn de belangrijkste conclusie(s)?
% - Perspectief: blijven er nog vragen open die in de toekomst nog kunnen
%    onderzocht worden? Wat is een mogelijk vervolg voor jouw onderzoek?
%
% LET OP! Een samenvatting is GEEN voorwoord!

Persoonlijke spraakassistenten als Amazon's Alexa, Apple's Siri of de Assistant van Google zijn aan een serieuze opmars bezig in Amerika. Tegenwoordig komen ze ook allemaal met een bijpassende smart speaker waar de spraakassistent is ingebouwd. In België blijft de populariteit nog uit, maar daar kan verandering in komen. Na de aankondiging van Google dat zijn persoonlijke assistent binnenkort een Belgische variant zal krijgen, sijpelen de eerste toepassingen van de grote bedrijven al binnen. Welk doel, naast winst maken, kan een toekomstige applicatie voor de Belgische variant nog hebben? Deze bachelorproef was eerst en vooral een zoektocht naar een doel die voor een bepaalde groep een meerwaarde is en waar op weinig weerstand wordt gebotst. Een bijzonder doelgroep bleek hiervoor niet de beste optie. Daarom is er gekozen om een applicatie te maken die de Vlaming kan helpen bij het verlenen van Eerste Hulp Bij Ongevallen. Veel informatie die nodig is om de applicatie te bouwen is af te leiden uit de recente mobiele applicatie die is gemaakt door het Rode Kruis en hetzelfde doel heeft.

Er is een literatuuronderzoek geschreven over de wereld van spraaktechnologie. Daaruit bleek dat spraakassistenten twee noodzakelijke functies hebben, namelijk spraakherkenning en spraaksynthese. Spraakherkenning of Speech-To-Text is gesproken taal omvormen naar voor de computer leesbare taal. Spraaksynthese is net het omgekeerde en is menselijke spraak gevormd door een computer.

Voor de applicatie kan ontwikkeld worden moet er een assistent gekozen worden waarvoor hij wordt gebouwd. Om er niet zomaar één uit te kiezen is er een vergelijkend onderzoek gevoerd naar de kwaliteit van de spraak en spraakherkenning van drie assistenten.
De spraakkwaliteit van de Engelstalige Google Assistant en Alexa en de Nederlandstalige Google Assistant werden met elkaar vergeleken a.d.h.v. beoordelingen van 30 vrijwilligers die EHBO-vragen hebben gesteld en de antwoorden van de assistenten hebben beluisterd. Voor dit deel van het onderzoek werd een kleine applicatie gemaakt voor elke assistent die als inhoud identiek dezelfde antwoorden gaven. De deelnemers konden een score geven op verstaanbaarheid, levendigheid, menselijkheid, tempo en emotionaliteit. De resultaten zijn barcharts die de drie assistenten vergelijken per eigenschap en boxplots die de vijf eigenschappen vergelijken per assistent. Allemaal op basis van de scores die zijn gegeven door de vrijwilligers.
Uit de resultaten van het eerste deel van het onderzoek worden enkele stellingen bevestigd. Deze tonen aan dat de Nederlandse Google Assistant significant lager scoort op spraakkwaliteit dan de andere twee assistenten. De Engelse assistenten kregen op elke eigenschap telkens een hogere score dan Google Assistant NL. Tussen de Engelse assistenten onderling was er niet veel verschil te merken. Google Assistent scoorde alleen op tempo significant hoger dan Alexa.

Het tweede deel van het onderzoek, het vergelijken van de kwaliteit van de assistenten hun spraakherkenning, was een moeilijkere taak die helaas niet echt in zijn opzet is geslaagd. Ondanks de maatregelen die zijn getroffen om de beïnvloeding van veranderlijke factoren te beperken, is het toch niet gelukt om statistisch correct onderzoek te voeren in het tweede deel. De bedoeling was om het aantal fouten die elke assistent maakt bij het herkennen van de gestelde vraag van de deelnemers te vergelijken. Zo werd er bijvoorbeeld wel aan gedacht om de gestelde vragen allemaal op te nemen, om ervoor te zorgen dat de assistenten later in een geluidsdichte studio identiek dezelfde dertig vragen te horen krijgen uit een speaker. Helaas werd er bijvoorbeeld niet verwacht dat het herkennen van een opgenomen audiofragment aanzienlijk zou verschillen van het herkennen van een rechtstreeks gestelde vraag. Dit is één van de redenen waarom de resultaten niet vergeleken worden. Er wordt wel gekeken naar interessante fouten die de assistenten hebben gemaakt. Uit de fouten die bij dit deel zijn gemaakt kunnen er lessen geleerd worden voor onderzoekers die in de toekomst werken met spraakherkenning.

Uiteindelijk is de keuze ondanks de resultaten toch gevallen op de Nederlandstalige Google Assistant. Dit komt omdat de Nederlandse taal een doorslaggevende factor is voor het verlenen van eerste hulp in België. Daarnaast wijst het erop dat Google als eerste assistant zijn intrek probeert te nemen in de Vlaamse woonkamers. De resultaten blijven bruikbaar bij het kiezen van een spraakassistent voor een eerste hulp-applicatie die wereldwijd wordt aangeboden in de Engelse taal. Als er enkel wordt afgegaan op de resultaten van het onderzoek naar de spraakkwaliteit dan is de Google Assistant de favoriet.

Als laatste is er een kort beeld geschetst van hoe een applicatie voor de Google Assistant werkt en hoe die werking kan verlopen bij de specifieke use case van het onderzoek. Het plan is om in de weken na het schrijven van deze bachelorproef een eerste versie te ontwikkelen van de eerstehulp-applicatie voor de Google Assistant. In de verdere toekomst kan deze versie getest worden. Feedback van gebruikers kan verwerkt worden en bij interesse van het Rode Kruis kan er mogelijks een samenwerking worden aangegaan.

%%---------- Nederlandse samenvatting -----------------------------------------
%
% TODO: Als je je bachelorproef in het Engels schrijft, moet je eerst een
% Nederlandse samenvatting invoegen. Haal daarvoor onderstaande code uit
% commentaar.
% Wie zijn bachelorproef in het Nederlands schrijft, kan dit negeren, de inhoud
% wordt niet in het document ingevoegd.

\IfLanguageName{english}{%
\selectlanguage{dutch}
\chapter*{Samenvatting}
\selectlanguage{english}
}{}

%%---------- Samenvatting -----------------------------------------------------
% De samenvatting in de hoofdtaal van het document



\chapter*{\IfLanguageName{dutch}{Samenvatting}{Abstract}}



%---------- Inhoudstafel ------------------------------------------------------
\pagestyle{empty} % No headers
\tableofcontents % Print the table of contents itself
\cleardoublepage % Forces the first chapter to start on an odd page so it's on the right
\pagestyle{fancy} % Print headers again

%---------- Lijst figuren, afkortingen, ... -----------------------------------

% Indien gewenst kan je hier een lijst van figuren/tabellen opgeven. Geef in
% dat geval je figuren/tabellen altijd een korte beschrijving:
%
%  \caption[korte beschrijving]{uitgebreide beschrijving}
\listoffigures
\listoftables

% Als je een lijst van afkortingen of termen wil toevoegen, dan hoort die
% hier thuis. Gebruik bijvoorbeeld de ``glossaries'' package.
% https://www.sharelatex.com/learn/Glossaries


%%---------- Kern -------------------------------------------------------------

%%=============================================================================
%% Inleiding
%%=============================================================================

\chapter{Inleiding}
\label{ch:inleiding}
Op 2 april 2019 kwam het Rode Kruis met het nieuws dat ze een mobiele applicatie hebben ontwikkeld voor het geven van eerste hulp bij ongevallen. Dit nieuws was dan ook de aanleiding om mijn onderzoek naar verschillende spraakgestuurde technologieën te koppelen aan het geven van EHBO. 

\section{Probleemstelling}
\label{sec:probleemstelling}

\section{Onderzoeksvraag}
\label{sec:onderzoeksvraag} 

\section{Onderzoeksdoelstelling}
\label{sec:onderzoeksdoelstelling}

\section{Opzet van deze bachelorproef}
\label{sec:opzet-bachelorproef}



\chapter{Stand van zaken}
\label{ch:stand-van-zaken}
\section{Spraakgestuurde technologie}
Er zijn enkele begrippen die met het thema te maken hebben, maar die niet hetzelfde omvatten. Taal- en spraaktechnologie is een verzamelnaam voor allerlei technieken waarmee de computer communiceert met zijn gebruiker door menselijke taal.\autocite{Taalunie2017} Het is de poging van de computer om de menselijke taal na te bootsen.

\subsection{Spraakherkenning}
Spraaktechnologie wordt vaak geassocieerd met spraakherkenning. Volgens \autocite{Rouse2016} is spraakherkenning de kunst van de computer om gesproken taal te identificeren en om te zetten naar voor de computer leesbare machinetaal.
Voor een computer succesvol spraak heeft omgevormd naar tekst, wordt er een lang en moeilijk proces doorlopen. Er wordt kort besproken wat de belangrijkste stappen zijn in het converteren van uitgesproken tekst naar tekst op een computerscherm.
\autocite{Vervoort2017}, \autocite{Geitgey2016} en \autocite{Woodford 2019} beschrijven hoe spraakherkenning in zijn werk gaat.
Wanneer een persoon iets uitspreekt, wordt zijn stem als geluidsgolven opgenomen door een microfoon. De analoge signalen worden omgezet naar digitale door een techniek genaamd sampling. Op vaste intervallen wordt de amplitude van de geluidsgolf gemeten en gedigitaliseerd. Het geluid is omgezet naar bits.

Uit het gedigitaliseerd geluid wordt geprobeerd om zo veel mogelijk ruis te filteren. Daarnaast wordt het ook naar een vast volume en een gelijke snelheid gebracht, omdat niet iedereen even snel en even luid spreekt. De uiteindelijke bedoeling is om een neuraal netwerk in te zetten om klanken en woorden te vormen uit het digitale geluid. Met een neuraal netwerk wordt de techniek bedoeld uit de IT-wereld die de werking van de hersenen gaat nabootsen om een computer zichzelf taken te leren.
Uit deze verkregen verzameling van getallen is het voor een neuraal netwerk nog steeds moeilijk om letters en woorden te herkennen. De bits worden eerst nog gegroepeerd in delen van ongeveer 20 milliseconden lang. Elke groep wordt dan opgesplitst in frequentiebanden waar telkens wordt nagegaan hoeveel energie er in vervat zit. Zo wordt een spectrogram gemaakt die een soort van vingerafdruk voorstelt van het geluid. In deze soort data kan een neuraal netwerk gemakkelijker patronen herkennen.
\begin{figure}[h]
    \includegraphics[width=0.7\linewidth]{img/spectogram}
    \caption{een voorbeeld van een spectrogram \autocite{Vervoort2017}}
    \label{fig:spectrogram}
\end{figure}
Dit is het moment waar het neurale netwerk zijn werk begint te doen. Hij gaat aan elk stukje data een spraakklank toekennen. Spraakklanken zijn de klanken die samen een taal vormt. Voor de Nederlandse taal bestaan er 40 verschillende spraakklanken. Klanken worden in woordenboeken fonetisch beschreven om duidelijk te maken hoe woorden worden uitgesproken. Het neurale netwerk gaat uit een fonetische lijst de spraakklank halen die de grootste waarschijnlijkheid heeft om correct bij een audiosignaal te horen. Je kunt nooit helemaal zeker zijn dat er een bepaalde klank is uitgesproken. Dit klinkt eenvoudiger dan het is. Elke klank wordt namelijk door elke mens op een andere manier uitgesproken. Door het netwerk te trainen met een grote hoeveelheid data, zal het beter leren omgaan met deze verscheidenheid.

De laatste stap is het omzetten van de klanken naar woorden. De klanken worden aan elkaar gelinkt tot woorden. Ook hierbij helpt trainingsdata om accurater woorden en zinnen te vormen. Als het netwerk bijvoorbeeld twijfelt tussen de woorden hallo, aloo en haylow, dan zal het waarschijnlijkheid kiezen voor 'hallo' omdat dit waarschijnlijk vaker voorkomt in de trainingsset. Daarnaast houdt het ook rekening met de waarschijnlijkheid dat een bepaald woord volgt op een woord. Ter illustratie, een netwerk zal door te trainen begrijpen dat de kans groter is dat het woord 'voorbeeld' gevolgd is op woorden zoals 'als, een of goed' dan op woorden zoals 'inktvis of tafel'. Voor neurale netwerken populair werden werd hiervoor een andere techniek gebruikt, namelijk het Hidden Markov Model. Hoe dit model precies werkt is buiten de scope van dit onderzoek.

\subsection{Spraaksynthese}
Een andere techniek, die net het omgekeerde is van spraakherkenning, is spraaksynthese. Volgens \autocite{Rouse2016} is spraaksynthese menselijke spraak dat is gevormd door een computer. Spraaksynthese is de basis voor elk Text-To-Speech systeem. Het wordt gebruikt om geschreven tekst om te zetten naar gesproken taal, geproduceerd door de computer.
Spraaksynthese is aanwezig in ons dagelijkse leven. In automatische telefoongesprekken, de luchthaven, gps-systemen, op de bus en natuurlijk in digitale assistenten. \autocite{Seijas2018} vertelt dat er twee soorten van methodes zijn voor Text-To-Speech. Concatenative TTS, waar korte audiofragmenten aaneengeschakeld worden, is daar één van. Het is goed verstaanbaar omdat de woorden zijn opgenomen in hoge kwaliteit, maar het klinkt niet natuurlijk. De andere methode, parametric TTS bedenkt de spraak gebaseerd op enkele parameters. Het haalt taalkundige kenmerken uit de tekst. Daarnaast extraheert het vocoder kenmerken die het corresponderende spraaksignaal representeert. Hier wordt niet verder op ingegaan omdat er nu een derde methode is opgedoken.
Deep Learning heeft ervoor gezorgd dat de Text-To-Speechsoftware geavanceerder werd. Eenvoudigweg is deep learning een benaming voor complexe neurale netwerken. 

--Hier nog uitleg over deep learning TTS--
--Hier nog uitleg over spraakhsynthese--
Ook iets vermelden over Lyrebird, waar je zelf al met je eigen stem text to speech kan doen in het Engels. iets over dat dat de nieuwe Deep Learning methode is.

Spraaktechnologie omvat naast deze twee begrippen nog meer. Denk maar aan spelling- en grammaticacontrole, spraak- en tekstanalyse, automatische vertalingen, enzovoort.

Spraakherkenning en spraaksynthese zijn nodig in het ontwikkelen van een Voice User Interface (VUI), of stemgestuurde gebruikersomgeving, waar de gebruiker de computer als het ware bedient met zijn stem in plaats van bijvoorbeeld een toetsenbord of aanrakingen. De computer moet gesproken taal van de gebruiker begrijpen (spraakherkenning) en moet een gepast antwoord teruggeven (spraaksynthese). Spraakassistenten zoals Alexa, Siri of Google Assistant zijn voorbeelden van VUI's.

Omdat een assistent de spraak kan omzetten naar tekst betekent dit nog niet dat hij begrijpt wat iemand heeft verteld. Assistenten worden ontwikkeld met Natural Language Processing. Het is een methode om ongestructureerde data gebaseerd op de natuurlijke taal te verwerken tot een vorm die de computer kan begrijpen. Het zorgt ervoor dat de betekenis of het doel wordt achterhaald van wat iemand zegt. NLP valt onder de noemer van Artificiële Intelligentie en het maakt gebruik van deep learning modellen. Modellen die getraind zijn om patronen te gaan herkennen in de menselijke taal door grote hoeveelheden aan data van bijvoorbeeld conversaties en berichten door te nemen. In principe is het vergelijkbaar met hoe een kind de taal leert, namelijk door naar voorbeelden te luisteren. \autocite{Rouse2017}
Volgens \autocite{Garbade2018} kan NLP voornamelijk onderverdeeld worden in twee niveaus, syntaxis en semantiek. Syntaxis is de grammatica van de tekst leren begrijpen. Het splitsen van zinnen of woorden en elk deel identificeren is één van de vele functies. Semantiek is de betekenis van de tekst leren begrijpen. Algoritmen worden gebruikt om bijvoorbeeld woorden te interpreteren en te classificeren als persoonsnaam of plaatsnaam.

\section{Spraakassistenten}
Een spraakassistent, ook wel een virtuele, persoonlijke of slimme assistent genoemd, voert taken uit via verbale instructies van een gebruiker. Het is vooral aanwezig in smartphones, maar het wordt ook geïntegreerd in smart speakers, auto's of wearables. Dit onderzoek vergelijkt twee van de meest prestigieuze assistenten, Google’s Assistant en Amazon’s Alexa. Daarnaast zijn ook Apple's Siri, Microsoft's Cortana en Samsung's Bixby bekende voorbeelden.

\subsection{De geschiedenis van spraakassistenten}
De slimme spraakassistenten zijn vandaag gekend bij het grote publiek. Ze zijn ingebouwd in onze smartphones en slimme luidsprekers. Steeds paraat om ons de vertragingen te melden op de weg, het weer te voorspellen voor morgen of onze favoriete muziek te spelen. Het is iets van deze tijd, maar toch hebben ze al een lang pad van tientallen jaren bewandeld. Dit is hoe het allemaal begon en hoe we zijn geëvolueerd naar de bekende assistenten van vandaag.

\subsubsection{Jaren 50 - 60}
De eerste systemen die ietwat leken op een spraakassistent waren gefocust op het louter herkennen van de menselijke spraak. In \autocite{Vox-Creative2019} wordt geschreven hoe in  de Bell Laboratories in 1952 het ``Audrey'' systeem werd ontwikkeld. Audrey begreep de getallen 0 tot 9 op voorwaarde dat de sprekers tussen elk getal een pauze lieten. In theorie kon het gebruikt worden om met de stem een telefoonnummer in te geven. Onder andere de kost en omvang van de machine was groot. Het intoetsen van de telefoonknoppen bleef efficiënter, dus het effectieve gebruik van Audrey bleef uit.

\autocite{IBM2011} onthulde in 1962 de ``Shoebox'', een machine die met spraakcommando's eenvoudige berekeningen kon uitvoeren. De uitvinder William C. Dersch demonstreerde voor televisie hoe het apparaat, zo groot als een schoendoos, naast de getallen 0 tot 9 ook zes woorden zoals plus en totaal kon herkennen.

\begin{figure}[h]
    \includegraphics[width=0.7\linewidth]{img/Shoebox}
    \caption{William C. Dersch’s Shoebox deed eenvoudige berekeningen met spraakcommando's \autocite{IBM2011}}
    \label{fig:shoebox}
\end{figure}

\subsubsection{Jaren 70 - 80}
Spraakherkenning in de jaren 70 werd vooral gekenmerkt door het departement voor defensie in de Verenigde Staten. Uit interesse voor spraakherkenning financierden ze een vijfjarig project over het thema. Volgens \autocite{Pinola2011} en \autocite{Kincaid2018} heeft dit geleid tot de ontwikkeling van Harpy in 1976. Harpy begreep 1011 woorden en kreeg vooral betekenis door haar efficiëntere zoekmethode, de ``Beam-search'', om logische zinnen te gaan herkennen.

In \autocite{Pinola2011} staat dat in de jaren 80 er een grote doorbraak kwam door de ontwikkeling van het hidden Markov model. Dit model gebruikt statistieken om een woord te herkennen in een onbekend geluid. Dit werd gedaan door het berekenen van de waarschijnlijkheid dat het onbekend geluid staat voor een bepaald woord. De woordenschat van de spraakherkenningssoftware bleef groeien tot een paar duizend woorden en had dankzij onder andere het hidden Markov model het potentieel om ongelimiteerd woorden te gaan herkennen.
Onder andere dankzij deze ontwikkelingen bleven ook de commerciële toepassingen niet uit. In 1987 kwam de Worlds of Wonder's doll Julie uit. Kinderen konden de pop trainen om te reageren op hun uitspraken. Dit staat zo beschreven in \autocite{Pinola2011}, waar je ook een reclamespot voor de pop kan bekijken. De technologie groeide snel, maar had wel een grote zwakte. De zin moest gedicteerd worden. Na elk woord werd dus een korte pauze verwacht.

\subsubsection{Jaren 90}
Volgens \autocite{Kincaid2018} kwam in de 90's automatische spraakherkenning in een eerste vorm zoals we het vandaag kennen. De doorbraak in die tijd heette Dragon. De eerste versie werd gelanceerd in 1990 onder de naam Dragon Dictate en had een woordenschat van 80 000 woorden. Daarnaast kon het iets nieuws, iets wat in de huidige spraakassistenten nog steeds gebruikt wordt, natural language processing. Zinnen moesten niet meer gedicteerd worden, maar Dragon kon oorspronkelijk 30 tot 40 woorden per minuut herkennen.

Volgens een artikel uit 1998 \autocite{Puri1998} is Dragon verantwoordelijk voor een doorbraak in spraakherkenningssoftware. De opvolger van de Dragon Dictate, Dragon NaturallySpeaking laat gebruikers spreken in een microfoon, aangesloten op de computer, en laat de woorden direct verschijnen op het computerscherm. Indien het een fout maakte, kon je het zelf corrigeren en kon de software leren uit zijn fouten. Het was ook de eerste spraakherkenningssoftware die toeliet om op een normale manier te praten.

\subsubsection{Van 2010 tot nu}
In \autocite{IBM2011} is te lezen hoe een mijlpaal werd bereikt door de Watson machine die won in Jeopardy. Watson was zo goed in taalverwerking dat hij 2 kampioenen in Jeopardy heeft verslaan live op televisie. Jeopardy is een Amerikaans spelprogramma waar de kandidaten het antwoord kregen en ze zelf de bijpassende vraag moesten geven. Watson was niet alleen goed in het samenstellen van correcte vragen, maar kon die ook telkens hardop uitspreken.

\begin{figure}[h]
    \includegraphics[width=0.7\linewidth]{img/WatsonJeopardy}
    \caption{Watson versloeg twee kampioenen in Jeopardy live op televisie \autocite{Markoff2011}}
    \label{fig:watson}
\end{figure}

\subsection{De spraakassistenten van nu}
--Kostprijs alexa en GA + ondersteuning talen meegeven--
Kort na deze gebeurtenis in 2011 werd Siri gebouwd in de Iphone 4S en werd zo de eerste spraakassistent voor het grote publiek uitgebracht. Siri is Apple's variant op de slimme spraakassistent en is tegenwoordig beschikbaar op meerdere apparaten met een IOS-besturingssysteem. Het grootste aandeel van de gebruikers kent Siri van op zijn Iphone, maar daarnaast is de assistent ook geïntegreerd in de Mac computer, de Apple Watch of de Apple TV. Ondertussen heeft het ook zijn eigen Smart Speaker, de HomePod.

Google gaf hierop een antwoord in 2012 door Google Now uit te brengen, de voorloper van de Google Assistant van vandaag. Volgens Google is de Google Assistant jouw eigen persoonlijke Google, die altijd bereid is om je te helpen wanneer je maar wilt. De Google Assistant bestond eerst onder de naam Google Now en was aanwezig in smartphones met een Android besturingssysteem. De Google Assistant van vandaag is te vinden in veel meer omgevingen. Smartphones, auto's, laptops, tablets, tv's, smartwatches en in hun eigen smart speaker, de Google Home. Deze speaker heeft ook een variant gekregen met een scherm, de Smart Display.

Tijdens de Microsoft BUILD conferentie in 2013 werd Cortana geïntroduceerd als de spraakassistent van Microsoft. Cortana is ontwikkeld voor onder andere Windows 10, Windows Phone, Xbox One en in de slimme speaker Invoke.

In 2015 kwam de eerste slimme luidspreker op de markt. De Echo van Amazon, voorzien met hun slimme spraakassistent, Alexa. Amazon is één van 's werelds grootste bedrijven in het online verkopen van goederen. Het grote verschil met Google is dat de assistent voor het eerst werd gebruikt in de Echo, Amazon's smart speaker.

\autocite{Lopez2018} besprak verschillende functionaliteiten van spraakassistenten Google Assistant, Alexa en Siri, op correctheid en natuurlijkheid bij 8 ondervraagden. In de administratieve categorie, zoals agendabeheer, to-dolijsten en alarmen kwam de Google Assistant als minst correct en minst natuurlijke assistent uit de bus, maar prijkt in de veelzijdige categorie (nieuws, weer, verkeer, woordbetekenissen, rekenen,, enz.) dan weer ver bovenaan op beide vlakken. Algemeen werd de Google Assistant als de meest natuurlijke ervaren, onder andere door de toon van de stem die verwondering, onzekerheid en vreugde uitte.
Voor \autocite{Tulshan2019} stelden 100 personen allerlei vragen aan voice assistants Google Assistant, Alexa, Siri en Cortana. Ze gaven telkens een score op spraakherkenning en contextueel inzicht. Google Assistant kwam als grote winnaar uit het onderzoek door 59,80 \% van de vragen te beantwoorden. Een verschil van 15,82 \% met Siri, die de op één na nauwkeurigste bleek in dit onderzoek. Google Assistant was vooral leider in categoriëen als reizen, mailing, navigatie, vertalingen en begreep volgens het onderzoek goed de verschillende variaties in de stemmen van de onderzochte personen.

Volgens \autocite{Lopez2018} is de Echo Dot de favoriete smart speaker als het aankomt op het aankopen van artikelen. Dit is geen verrassing omdat hij oorspronkelijk ontworpen is om te winkelen en zelfs de enige spraakassistent is waarmee je online kan shoppen. In \autocite{Tulshan2019} bleek Alexa de minst nauwkeurige assistent te zijn met 7,91 \% nauwkeurigheid.

Google spendeert veel middelen aan zijn assistent. Het heeft indruk gemaakt tijdens de recente Google I/O conferentie van dinsdag 7 mei 2019, waar ze onverwachts nieuws brachten. De volgende versie van de Google Assistant zal namelijk opvallend veel sneller gaan omdat ze de AI-modellen die verantwoordelijk zijn voor NLP offline beschikbaar hebben gemaakt. Dat wilt zeggen dat een commando van een gebruiker niet meer helemaal naar een server in Amerika moet transporteren om ervoor te zorgen dat de assistent het kan begrijpen. Vanaf de volgende versie zal deze logica afgehandeld worden door uw toestel zelf, omdat Google erin geslaagd is het geheugen van die modellen zo te reduceren dat het kan opgeslagen worden op uw apparaat. Het belooft dus dat binnenkort de gebruiker na het stellen van een vraag amper nog zal moeten wachten op een antwoord.

VRT NWS meldt op 28 mei dat de Belgische variant van de Google Assistant wordt gelanceerd. \autocite{Belghmidi2019} Voorlopig nog alleen met een Nederlands accent. Wat er wel bijkomt is de samenwerking met verschillende Belgische bedrijven. Zo kan iedereen binnenkort via de Google Assistant naar het radionieuws van VRT NWS luisteren, aan de NMBS vragen wanneer de volgende trein komt of artikelen van de Colruyt toevoegen aan zijn lijstje.

Slimme spraakassistenten worden alleen maar slimmer. \autocite{Brandt2018} heeft onderzocht hoe hoog het intelligentieniveau is van 4 slimme assistenten, namelijk Google Assistant, Microsoft Cortana, Amazon Alexa en Apple Siri in 2017 en 2018. De geanalyseerde gegevens zijn de antwoorden van de assistenten op 5000 algemene vragen. De beste prestatie werd verricht door Google Assistant die op 77,2 procent van de vragen een antwoord kon bieden, waarvan 95 procent correct. Bij alle assistenten zie je een verhoging van de intelligentie in vergelijking met het vorige jaar.

\begin{figure}[h]
    \includegraphics[width=0.7\linewidth]{img/SmartAssistantsAreGettingSmarter}
    \caption{Hoe hoog ligt het intelligentieniveau bij slimme spraakassistenten \autocite{Brandt2018}}
    \label{fig:smartassistantsaregettingsmarter}
\end{figure}

--Hier nog verder doen met overige bronnen over assistenten vandaag--

Als het gaat over de spraakkwaliteit, ook meer uitleggen wat er precies is beoordeeld.
-> De Text-to-speech of speech synthesis, bij Google Assistant met WaveNet: https://towardsdatascience.com/wavenet-google-assistants-voice-synthesizer-a168e9af13b1

\subsection{Hoe werkt een spraakassistent}
--De flow uitleggen van de gebruiker die iets vraagt tot de assistent die een antwoord heeft gegeven--

\section{Bestaande eerste hulpapplicaties}
\subsection{De Vlaamse EHBO-app van het Rode Kruis}
Op 2 april ’19 kwam het Rode Kruis met het nieuws dat ze een app hebben ontwikkeld die kan helpen bij het geven van eerste hulp bij ongevallen. 80 procent van de Vlamingen weet niet wat hij moet doen als een nabije persoon begint te stikken, een hartstilstand krijgt of hevig begint te bloeden. Uit angst om iets fouts te doen, gebeurt er dan ook vaak niks. Met de app willen ze zoveel mogelijk mensen in staat stellen om hulp te verlenen. \autocite{Decroubele2019}

Het Rode Kruis benadrukt dat de applicatie de opleiding niet kan vervangen, maar dat het hulp kan bieden bij het geven van eerste hulp.

In de applicatie zijn er drie grote onderdelen, eerste hulp verlenen, eerste hulp leren en een AED-toestel vinden in de buurt. Er zijn ook nog enkele opties die je naar de website van het Rode Kruis brengen om informatie te verkrijgen over het geven van bloed of plasma, het doen van een gift, het volgen van een opleiding of het aanmelden als vrijwilliger.

Als je eerste hulp wilt verlenen kun je uit het overzicht een onderwerp over eerste hulp kiezen, waarna je informatie krijgt over wat je moet vaststellen en wat je nodig hebt. Daarnaast geeft de app ook een stappenplan van instructies wat je moet doen. De levensbedreigende situaties staan helemaal bovenaan en zijn voorzien van extra ingesproken instructies.

Als je eerste hulp wilt leren kun je eerst een onderwerp kiezen. Voorbeelden zijn een beroerte of alcoholvergiftiging. Daarna krijg je over het onderwerp vragen \& antwoorden, informatieteksten en video's. Per leerdeel krijg je een quiz die je moet oplossen om bepaalde badges te verdienen.

Wanneer iemand in uw omgeving een hartstilstand krijgt dan kun je met de applicatie een kaart openen waar AED-toestellen staan op gesitueerd. Je kunt er ook een nieuwe AED melden of meer informatie lezen.


\subsection{Andere EHBO-applicaties}
Nederland heeft al langer een mobiele EHBO-applicatie. Deze verschilt niet zo veel met de Belgische versie. Ze heeft wel een zoekfunctie om sneller de instructies voor uw ongeval te vinden. Je kan er ook EHBO-kits en cursussen bestellen in de webshop.

--Nog iets over Engelstalige applicaties--

%%=============================================================================
%% Methodologie
%%=============================================================================

\chapter{Methodologie}
\label{ch:methodologie}

\section{Vergelijking van stemgestuurde technologieën}
\label{sec:vergelijking van stemgestuurde technologieën}

\subsection{Algemeen}
\label{sec:algemeen}

Waarom voor deze technologiëen gekozen om te onderzoeken. Waarom geen Siri, Mycroft en Microsft Cortana.

\begin{center}
    \begin{tabular}{ | l | l | l | l |}
        \hline
        Assistant & Smart speaker & Prijs & Nederlandse ondersteuning\\ \hline
        Google Assistant & Google Home (Mini) & +-60 EUR en +-150 EUR & Ja \\ \hline
        Amazon Alexa & Amazon Echo (dot) & +-74 EUR en +-122 EUR & Neen \\ \hline
        Mycroft & Mark I & 132,55 EUR & Neen \\ \hline
    \end{tabular}
\end{center}

\subsection{Gebruik van de Assistant}
\label{sec:gebruik van de assistant}
functionaliteiten die belangrijk zijn:
automatisch een telefoontje plegen: Google Assistant gaat niet

Wat is de response time? Boxplot

Welke middelen zijn er nodig om het te gebruiken?

Hoe uitgebreid is de functionaliteit?

Hoe vaak wordt hij begrepen bij een lange vraag? (op hetzelfde smartphone device) succesratio

Hoe verstaanbaar vindt een gebruiker het antwoord? Score op correctheid en natuurlijkheid

Tot welke afstand blijft de assistant je begrijpen? (op hetzelfde smartphone device)

\subsection{Ontwikkelen van een applicatie}
\label{sec:ontwikkelen van een applicatie}
Welke middelen kan je gebruiken om het te ontwikkelen?

Hoe uitgebreid is de documentatie?

Welke programmeertalen worden aangeboden?

Kan de applicatie automatisch een nummer bellen?

\subsection{Vergelijking van de verschillende technologieën}
\label{sec:vergelijking van de onderzochte technologieën}
\textbf{Response Time}
Resultaat: Grafiek

\textbf{Middelen}
Resultaat: Tabel

\textbf{Functionaliteit}
Resultaat: Grafiek

\textbf{Begrijpen}
Resultaat: Grafiek

\textbf{Ontwikkelen van een eigen applicatie}
Resultaat: Tabel

\section{Orthopedagogisch onderzoek}
\label{Orthopedagogisch onderzoek}
\subsection{De zoektocht naar een gepaste doelgroep}
\label{De zoektocht naar een gepaste doelgroep}

\subsubsection{Ondersteuning van de begeleider in de jeugdzorg}
\label{ondersteuning van de begeleider in de jeugdzorg}
De aanleiding van dit onderzoek was een onderzoek van -citeer onderzoek mr. Buysse- over faciliterende IT bij individuele begeleidingsgesprekken in de jeugdzorg. Uit het onderzoek ontstond er twijfel over het gebruik van een spraakassistent als ondersteuning bij het individuele begeleidingsgesprek tussen de persoonlijke begeleider en het kind. Door de twijfel werd er dan ook beslist om niet verder te gaan met spraaktechnologie, maar met andere digitale tools.

Om er toch zeker van te zijn dat er geen mogelijkheden waren, ontstond deze bachelorproef. Echter, na een eerste gesprek met Iris Storme, docent orthopedagogie binnen Hogeschool Gent en tevens mede-researcher van -citeer onderzoek mr. Buysse- werden voor mij de beweegredenen voor het afkeuren van spraaktechnologie binnen hun onderzoek snel duidelijk. 

Als je denkt aan mensen met een visuele of fysieke beperking komen er snel mogelijkheden naar boven. Denk maar aan het controleren van apparaten met een eenvoudig stemcommando. Deze personen kunnen technologie als een mogelijke oplossing zien, waardoor zij, en de begeleiders, dit gemakkelijker kunnen omarmen.
Daartegenover staat de bijzondere jeugdzorg, waar de spraakassistent eerder ondersteuning zou bieden in de emotionele problematiek en de jongeren net hun façade nodig hebben om overeind te blijven. Deze doelgroep stelt zich niet zo graag kwetsbaar op en ervaart het praten over gevoelens eerder als een drempel. De bijzondere jeugdzorg lijkt op het eerste zicht een minder relevante doelgroep.

Dit werd allemaal vastgesteld door het intuïtieve gevoel van de ondervraagde. Dit was voor mij persoonlijk voldoende om na te gaan denken over een nieuwe doelgroep.

\subsubsection{Ondersteuning van personen met het syndroom van Down}
\label{ondersteuning van personen met het syndroom van Down}
Ik wijzigde mijn doelgroep naar personen die geboren zijn met trisomie 21, ook wel het syndroom van Down genoemd. Uit een eerste opzoeking stelde ik de volgende mogelijkheden.

Personen met het Downsyndroom worden geboren met een verstandelijke beperking. Er kan gekeken worden naar welke noden uit die verstandelijke beperking vloeien, bijvoorbeeld moeite met rekenen, en hoe een spraakassistent hier ondersteuning kan bieden. Dit kan ook veel verder gaan als in vb. het helpen met zelfstandig wonen.

Daarnaast zijn er ook mogelijke bijkomende aandoeningen zoals een minder goed geheugen, coeliakie, slaapapneu, oogafwijkingen of een gedragsstoornis. Hier kan spraakassistentie mogelijks ook ondersteuning in bieden. Ik denk aan bijvoorbeeld interactieve activiteiten voor het stimuleren van de motoriek, het geheugen of het spraakvermogen, helpen herinneren aan belangrijke taken, helpen herinneren aan wat ze wel of niet mogen eten, stimuleren van een vast slaappatroon, enzovoort.

--Bronnen vermelden--

Dit waren nog maar losse ideeën die ontstonden uit een eerste verkenning over personen met het gendefect. Het werd duidelijk dat hier zeker mogelijkheden waren, dus was de volgende stap om hierin gaan te verdiepen door interviews af te nemen van mensen die een persoonlijke ervaring hebben met deze doelgroep.

Echter, mijn eerste aanvraag voor een interview aan iemand van wie haar dochter geboren is met trisomie 21 stootte direct op weerstand. De respons die ik kreeg ging over het gevaar van het veralgemenen. Het is een complexe materie omdat het gaat over een combinatie van samenkomende symptomen en kenmerken. De effecten van het gendefect kunnen in verschillende gradaties voorkomen, waardoor elke persoon die het gendefect bezit uniek moet bekeken worden.
%%=============================================================================
%% resultaten vergelijkend onderzoek
%%=============================================================================

\chapter{Resultaten vergelijkend onderzoek}
\label{ch:Resultaten vergelijkend onderzoek}
Wie in detail wilt bekijken hoe de resultaten zijn bekomen, kan meer informatie vinden bij \ref{appendix:stappenplan analyse}.

\section{Vergelijking van de assistenten in spraakkwaliteit}
\label{s:Vergelijking van de assistenten in spraakkwaliteit}

\subsection{Vergelijking van de assistenten per eigenschap}
\label{ss:Vergelijking van de assistenten per eigenschap}

\begin{figure}[h]
    \includegraphics[width=0.9\linewidth]{../onderzoek/onderzoeksresultaten/vergelijking_assistenten_per_eigenschap/barplot/barplot_score_emotionaliteit}
    \caption{De score die de deelnemers hebben gegeven op de emotionaliteit van de assistenten}
    \label{fig:barplot-emotionaliteit}
\end{figure}

\begin{figure}[h]
    \includegraphics[width=0.9\linewidth]{../onderzoek/onderzoeksresultaten/vergelijking_assistenten_per_eigenschap/barplot/barplot_score_levendigheid}
    \caption{De score die de deelnemers hebben gegeven op de levendigheid van de assistenten}
    \label{fig:barplot-levendigheid}
\end{figure}

\begin{figure}[h]
    \includegraphics[width=0.9\linewidth]{../onderzoek/onderzoeksresultaten/vergelijking_assistenten_per_eigenschap/barplot/barplot_score_menselijkheid}
    \caption{De score die de deelnemers hebben gegeven op de menselijkheid van de assistenten}
    \label{fig:barplot-menselijkheid}
\end{figure}

\begin{figure}[h]
    \includegraphics[width=0.9\linewidth]{../onderzoek/onderzoeksresultaten/vergelijking_assistenten_per_eigenschap/barplot/barplot_score_tempo}
    \caption{De score die de deelnemers hebben gegeven op het tempo van de assistenten}
    \label{fig:barplot-tempo}
\end{figure}

\begin{figure}[h]
    \includegraphics[width=0.9\linewidth]{../onderzoek/onderzoeksresultaten/vergelijking_assistenten_per_eigenschap/barplot/barplot_score_verstaanbaarheid}
    \caption{De score die de deelnemers hebben gegeven op de verstaanbaarheid van de assistenten}
    \label{fig:barplot-verstaanbaarheid}
\end{figure}

\begin{figure}[h]
    \includegraphics[width=0.9\linewidth]{../onderzoek/onderzoeksresultaten/vergelijking_assistenten_per_eigenschap/table_mean_sd_scores}
    \caption{De gemiddelde score en standaardafwijking van alle assistenten hun eigenschappen gesorteerd van hoog naar laag.}
    \label{fig:table-mean-sd-scores}
\end{figure}

\subsection{Vergelijking van de eigenschappen per assistent}
--Hier boxplots--

De resultaten van de uitgevoerde t-testen zijn te vinden onder de map onderzoek/onderzoeksresultaten in de repository beschreven in \ref{s:verwijzing naar repository}.

De p-waarde ligt bij elke test duidelijk onder het significantieniveau van 0.05 dus kunnen we de nulhypothese verwerpen. De volgende stellingen zijn statistisch verantwoord.
\begin{itemize}
    \item Alexa scoort significant hoger op emotionaliteit dan GA NL.
    \item GA scoort significant hoger op emotionaliteit dan GA NL.
    \item Alexa scoort hoger op levendigheid dan GA NL.
    \item GA scoort hoger op levendigheid dan GA NL.
    \item Alexa scoort hoger op menselijkheid dan GA NL.
    \item GA scoort hoger op menselijkheid dan GA NL.
    \item GA scoort hoger op tempo dan Alexa.
    \item Alexa scoort hoger op tempo dan GA NL.
    \item GA scoort hoger op verstaanbaarheid dan GA NL.
\end{itemize}

\section{Vergelijking van de assistenten in spraakherkenning}
\subsection{Een moeilijk te voeren onderzoek}
Uitleg over hoe moeilijk het was om dit onderzoek correct te voeren ondanks de genomen maatregelen beschreven in de methodologie.

Een eerste bemerking bij het onderzoek is dat elke fout gelijk meetelt. Stel dat twee assistenten de uitspraak 'help me with a burn' omvormen naar tekst. De ene assistent begrijpt de vraag als 'help us for a burn' en de andere als 'tell me with a good'. Alhoewel het gevoel zegt dat de eerste assistent de vraag beter heeft begrepen dan de tweede hebben ze toch allebei evenveel fouten gemaakt. de eerste assistent mist de woorden 'me' en 'with', terwijl de tweede assistent de woorden 'help' en 'burn' mist. Om de correctheid van een gevormde zin beter te interpreteren zou aan elk woord een soort van hoofdzakelijkheid voor het begrijpen van de zin moeten toegekend worden. Er bestaan echter geen vaste regels om dit aan woorden toe te wijzen.

Ondanks dat er verschillende maatregelen zijn genomen om ervoor te zorgen dat elke assistent identiek dezelfde vraag krijgt, is deze opzet toch niet helemaal geslaagd. Alexa neemt elke conversatie op en bewaart ze in uw geschiedenis. Door enkele conversaties van tijdens het onderzoek te beluisteren is er opgemerkt dat hier en daar gekraak aanwezig is. Het gekraak was tijdens het onderzoek niet hoorbaar, waardoor de oorzaak waarschijnlijk bij de microfoon van de smartphone ligt. Het onderzoek is niet correct gevoerd omdat de sterkte van het gekraak kan variëren en mogelijks invloed heeft op de prestaties van de Speech-To-Text-functie van een assistent.


%%=============================================================================
%% De eerste hulp assistent
%%=============================================================================

\chapter{De eerste hulp assistent}
\label{ch:De eerste hulp assistent}

\section{De functionaliteiten}
\label{De functionaliteiten}

\section{Persona}
\label{Persona}




% Voeg hier je eigen hoofdstukken toe die de ``corpus'' van je bachelorproef
% vormen. De structuur en titels hangen af van je eigen onderzoek. Je kan bv.
% elke fase in je onderzoek in een apart hoofdstuk bespreken.

%\input{...}
%\input{...}
%...
%%=============================================================================
%% Conclusie
%%=============================================================================

\chapter{Conclusie}
\label{ch:conclusie}

Uit het onderzoek is gebleken dat de algemene spraakkwaliteit van de Engelstalige Alexa en Google Assistant hoger liggen dan die van de Nederlandstalige Google Assistant. Bij elke eigenschap scoorden ze significant hoger dan de Nederlandstalige assistent, met uitzondering van de verstaanbaarheid van Alexa. Voor deze eigenschap kan niet bewezen worden dat de score bij Alexa significant hoger ligt dan bij Google Assistant NL.
Omdat de applicatie moet helpen met eerste hulp verlenen in België is het absoluut nodig dat de assistent Nederlands kan. Google Assistant en Siri zijn de enige twee spraakassistenten die op dit moment een Nederlandse versie hebben. Omdat een ontwikkelaar Siri enkel kan integreren in een bestaande mobiele applicatie gaat de voorkeur eerder naar Google Assistant. Daarnaast is er ook nog het nieuws dat er binnenkort een Vlaamse versie komt van de Google Assistant. Google doet dus uitschijnen dat ze met hun Google Home als eerste willen proberen intrekken in de Vlaamse woonkamers. Deze factoren hebben meegedragen aan het antwoord dat de Google Assistant het meest geschikt is om te helpen bij het verlenen van eerste hulp bij ongevallen. We kunnen besluiten dat er uiteindelijk voor de keuze van de meest geschikte spraakassistent geen rekening is gehouden met de resultaten van het vergelijkend onderzoek.

Dit geldt echter alleen voor België en voor nu. Er kan ook op wereldwijd niveau gekeken worden naar het ontwikkelen van een spraakgestuurde eerstehulp-applicatie. In dit geval kan er wel rekening worden gehouden met de resultaten van het vergelijkend onderzoek voor de landen waar vooral Engels wordt gesproken. De twee Engelstalige assistenten verschillen echter maar aanzienlijk op één eigenschap, namelijk het tempo. De participanten vinden dat Google Assistant op een gepaster tempo praat dan Alexa. Een niet gepast tempo is een tempo dat ervaren wordt als te snel of te traag. Als de meest geschikte spraakassistent alleen o.b.v. deze resultaten wordt gekozen, dan komt Google Assistant opnieuw als winnaar uit de bus komen. De kans bestaat ook dat Alexa of andere spraakassistenten later met een Nederlandse of Vlaamse versie op de markt komen. Het zou interessant zijn om dan dezelfde vergelijking te maken tussen Nederlandstalige assistenten. 

Een correct onderzoek naar de kwaliteit van spraakherkenning verloopt moeilijker dan gedacht. Er is besloten om het aantal fouten die elke assistent heeft gemaakt bij het omvormen van spraak naar tekst niet te tonen of te vergelijken. Waarom dit deel van het onderzoek moeilijk uit te voeren is op een correcte manier, staat beschreven in \ref{ss:een moeilijk te voeren onderzoek}. Er kunnen lessen uit geleerd worden voor toekomstige onderzoekers die gaan werken in hetzelfde onderzoeksveld.

Na het schrijven van deze bachelorproef is er het plan om de eerste stappen te zetten in het ontwikkelen van de applicatie. De meest geschikte spraakassistent staat vast en de structuur is getekend. De eerste versie kan ingezet worden om feedback te verzamelen van testgebruikers. Daarnaast kan het Rode Kruis gecontacteerd worden om te polsen wat ze daar van de applicatie vinden. Mogelijks zijn ze geïnteresseerd in een verdere samenwerking voor het ontwikkelen van de applicatie.

Dat de spraaktechnologie in een stroomversnelling is beland is duidelijk. In Amerika hebben 1 op 3 inwoners al aankopen gedaan met hun smart speaker. Alexa van Amazon heeft zichzelf daar ontpopt als favoriet door onder andere de koppeling aan hun webshop. Een sterkte die ze hier nog niet kunnen inzetten door het gebrek aan een Nederlandse versie. Daardoor ziet Google zijn kans en doet ze haar best om hier te groeien als de nummer één in spraakassistentie met Google Assistant en de slimme Smart Speaker Google Home. Spraakbesturing heeft veel potentieel, zeker in combinatie met the internet of things. Zonder zelfs te moeten opstaan kun je de rolluiken openen, een film starten op netflix of de lampen dimmen. Het klinkt aanlokkelijk. Zal de Smart Speaker eerst een hebbeding worden voor de early adapters en vooral, zal er met enkele jaren sprake zijn van een derde digitale golf, na het internet en de smartphone?


%%=============================================================================
%% Bijlagen
%%=============================================================================

\appendix

%%---------- Onderzoeksvoorstel -----------------------------------------------

\chapter{Onderzoeksvoorstel}
\label{apendix:onderzoeksvoorstel}
Het onderwerp van deze bachelorproef is gebaseerd op een onderzoeksvoorstel dat vooraf werd beoordeeld door de promotor. Dat voorstel is opgenomen in deze bijlage.

% Verwijzing naar het bestand met de inhoud van het onderzoeksvoorstel
%---------- Inleiding ---------------------------------------------------------

\section{Introductie} % The \section*{} command stops section numbering
\label{sec:introductie}
Kinderen in jeugdcentra hebben een individuele begeleider nodig. De begeleider is verantwoordelijk voor het kind de aandacht, begeleiding en zorg op maat te geven die hij of zij nodig heeft. De begeleider is de vertrouwensfiguur voor het kind.
Een probleem dat zich voordoet is dat de begeleider alleen aanwezig is tijdens zijn werkuren. De vertrouwenspersoon kan voor een periode wegvallen en tijdens deze periode is het kind zijn steunfiguur kwijt.

Spraakgestuurde technologie is aan een opmars bezig. In de Verenigde Staten hebben één op de vijf volwassenen toegang tot een smart speaker met stemassistent.~\autocite{Passies2018} In België is het gebruik van een slimme luidspreker nog niet van de grond. Dit kan sinds de komst van de Nederlandse versie van Google Assistent wel eens gaan veranderen. Ook Google's slimme luidspreker, Google Home, kwam eind oktober voor het eerst op de Nederlandse markt.~\autocite{Haenen2018} Het apparaat bevat de nieuwe Google Assistent en is daardoor de eerste smart speaker die Nederlands begrijpt. Consumenten hebben een heel andere band met een voice assistant dan met een ander apparaat. Ze spreken over hun apparaat alsof het een mens is.\autocite{Schueler2018} Een gesprek voeren met een apparaat kan zorgen voor een persoonlijke band, iets wat bij de bekende smartphone ontbreekt.

%Iets over dat ik gemotiveerd ben omdat het gloednieuw is dus heel veel mogelijkheden ontstaan enzoo allemaal nog nieuw he.
Het commerciële gebruik van deze technologie in ons land is dus gloednieuw. Het onderzoek over wat we met deze technologie allemaal kunnen verwezenlijken, kan voor veel mogelijkheden zorgen.

%Hier beginnen over de doelstelling van het onderzoek en onderzoeksvragen
Doel van het toegepast onderzoek is om na te gaan of stemgestuurde technologie gebruikt kan worden om kinderen in jeugdcentra te ondersteunen. Het is niet de bedoeling om een apparaat te ontwikkelen die de begeleider in de toekomst zal vervangen. Het kan het kind helpen ondersteunen wanneer zijn individuele begeleider niet aanwezig is in het jeugdcentrum.
Het onderzoek bevat volgende onderzoeksvraag: Hoe kan stemgestuurde technologie helpen met de begeleiding van kinderen in jeugdcentra bij afwezigheid van de individuele begeleider? 
De onderzoeksvraag bestaat uit twee deelvragen, elk binnen zijn aparte domein. Op orthopedagogisch vlak luidt de vraag: Welke noden zijn er om een stemgestuurde applicatie te ontwikkelen? Vertrekkende vanuit deelvraag 1 gaat het binnen het technologische domein over: Welke spraaktechnologie biedt de meeste mogelijkheid voor een goede oplossing?

%---------- Stand van zaken ---------------------------------------------------

\section{State-of-the-art}
\label{sec:state-of-the-art}

De laatste jaren wordt er meer en meer onderzoek gedaan naar hoe IT en zorg kunnen samenwerken. Zo kwam de overheid met e-health, een elektronisch platform waar alle betrokkenen in de volksgezondheid gegevens kunnen uitwisselen.

Een belangrijk begrip waar onderzoek naar gedaan wordt is Blended Care. Geestelijke gezondheidszorg, ondersteund door IT. Het beste proberen gebruiken van beide werelden. De patiënt krijgt online een behandeling, maar wordt daarnaast ook nog steeds ondersteund door een begeleider. De mix van online en face-to-face therapie heeft al bewezen veel voordelen te hebben. Uit het SROI-verslag van \textcite{Stil2016} concludeert men dat gemiddeld over vijf jaar, de investeringen voor Blended Care 2,2 keer dit bedrag aan maatschappelijke baten opleveren. De cliënt krijgt de mogelijkheid om zelf aan zijn geestelijke gezondheid te werken tussen sessies met de begeleider door. Wat er voor zorgt dat de cliënt vertrouwen krijgt in het zelfstandig omgaan met zijn gezondheid.~\autocite{Wentzel2016} Uit het net vermelde onderzoek blijkt ook dat er wel nog meer onderzoek nodig is om te bepalen welke precieze mix de voorkeur krijgt van cliënt en begeleider in bepaalde situaties.

Eén van de grootste behaalde projecten die technologie in de zorgsector toepast is Zora, de zorgrobot. Een slimme robot die nu wordt ingezet in verschillende zorginstellingen. Het wordt beklemtoond dat het niet de bedoeling is om mensen in de zorg te vervangen, maar te begeleiden.~\autocite{Grypdonck2015} Er verschijnen veel positieve berichten over Zora en haar functionaliteiten, maar er is wel zeker één groot hekelpunt, de kostprijs. Die ligt namelijk rond de vijftienduizend euro.~\autocite{Jongejan2016}

Hoewel het dus bewezen is dat zorg en technologie samengaan, is er nog geen specifiek onderzoek gedaan naar het gebruik van een smart speaker in de jeugdzorg. Dit nicheproduct kan mogelijks toegevoegde waarde bieden aan het grotere geheel. Zo is ook al bewezen dat IT ervoor zorgt dat de drempel om de stap naar hulp te zetten lager wordt door de anonimiteit die ermee gepaard gaat.~\autocite{Stil2016} Het gebruik van een smart speaker biedt veel mogelijkheden om de zorgsector te verbeteren.

Een andere vraag is of mensen klaar zijn om gebruik te maken van deze technologie. De jeugd staat meer open voor het gebruik van nieuwe technologische middelen, maar ook de begeleiders moeten hier mee akkoord gaan. De heer Buysse, mijn promotor voor deze bachelorproef, is aan een project bezig over faciliterende IT bij individuele begeleidingsgesprekken in de jeugdzorg. Uit navraag bij ex-cliënten en begeleiders bleek dat IT geen oplossing is om het gesprek te vervangen. Vandaar ook dit specifieke onderzoek naar stemgestuurde technologie als ondersteuning bij afwezigheid van de individuele begeleider.

% Voor literatuurverwijzingen zijn er twee belangrijke commando's:
% \autocite{KEY} => (Auteur, jaartal) Gebruik dit als de naam van de auteur
%   geen onderdeel is van de zin.
% \textcite{KEY} => Auteur (jaartal)  Gebruik dit als de auteursnaam wel een
%   functie heeft in de zin (bv. ``Uit onderzoek door Doll & Hill (1954) bleek
%   ...'')

%---------- Methodologie ------------------------------------------------------
\section{Methodologie}
\label{sec:methodologie}
Eerst en vooral zal er parallel onderzoek gedaan worden naar een antwoord op de twee deelvragen van het onderzoek.
Op orthopedagogisch domein wordt er aan veldonderzoek gedaan. Er wordt zoveel mogelijk gekeken naar wat de vraag is bij begeleider en kind. Wat zijn de vereisten voor de applicatie? Dit kan gedaan worden door kwalitatief onderzoek in de vorm van interviews en/of enquêtes. Die kunnen afgenomen worden bij alle betrokkenen. Dit kan ver gaan, maar er zal vooral gefocust worden op de directe doelgroep, de begeleiders en kinderen in jeugdcentra. Er kunnen ook gegevens verzameld worden door te observeren.
%Literatuurstudie over het onderwerp? Gaat er wel wat te vinden zijn?
%andere betrokkenen: ortho's

Op technologisch domein wordt er kennis verworven over de verschillende mogelijkheden van spraakgestuurde technologie. Een vergelijkende studie die de voor- en nadelen van de bestaande technologiën afweegt, zal beslissen welke technologie er gekozen wordt. Met de gekozen optie zal er uiteindelijk verder gewerkt worden.
%En literatuurstudie vermelden?

Nadat beide evaluaties worden gematcht, zal er een proof of concept opgesteld worden die zal beslissen of de applicatieontwikkeling wel of niet wordt gestart. Als de applicatie mag ontwikkeld worden zal er dikwijls naar feedback van de directe betrokkenen gepolst worden. De bedoeling is om dan tegen het einde van het onderzoek een eerste versie van de applicatie te ontwikkelen waarop later kan worden voortgebouwd.
%Wat als er beslist wordt dat er beter geen applicatie wordt gemaakt?

%---------- Verwachte resultaten ----------------------------------------------
\section{Verwachte resultaten}
\label{sec:verwachte_resultaten}

Concrete resultaten zijn moeilijk te voorspellen aangezien er geen metingen en simulaties worden gedaan in het onderzoek. Uit interviews en observaties kunnen er onvoorspelbare en uiteenlopende resultaten ontstaan waar de onderzoeker zelf nooit zou kunnen opkomen.

Er kan wel al nagedacht worden over welke noden de betrokkenen kunnen hebben. Zo kan er nood zijn aan een smart speaker die een gesprek kan aangaan met het kind wanneer hij/zij er naar vraagt. Indien gewenst kan het gesprek opgenomen worden zodat dit later kan beluisterd worden door de begeleider. Het kind geeft beter eerst de toestemming om het gesprek op te nemen zodat het apparaat het vertrouwen van het kind niet schaadt.
Het kan een optie zijn om het apparaat te doen reageren op crisismomenten van een kind. Inspelen op het moment dat het kind een crisismoment beleeft, kan een belangrijke verantwoordelijkheid van het apparaat worden.
Het kan goed zijn dat het apparaat gegevens uit het verleden kan ophalen om het kind zo goed mogelijk te ondersteunen.
De technologie die zal gebruikt worden lijkt vooral te neigen naar de Google Home omdat dit voorlopig de eerste Nederlandstalige Smart Speaker is op de markt.

%---------- Verwachte conclusies ----------------------------------------------
\section{Verwachte conclusies}
\label{sec:verwachte_conclusies}

Er wordt verwacht dat uit de interviews, enquêtes en observaties ideeën voortvloeien die de functionaliteiten van de applicatie zullen bepalen. Het zal vast en zeker een uitdaging worden om het apparaat als een vertrouwenspersoon te doen fungeren voor het kind.
Kritiek kan een hindernis worden tijdens het onderzoek.
Orthopedagogen kunnen ervan overtuigd zijn dat een vertrouwensband alleen kan ontstaan tussen mensen. Dergelijke personen kunnen weigerachtig staan tegenover het gebruik van technologie in hun domein.
Dat er tegenstanders zijn bewijst ook het bestaan van de website www.zorgictzorgen.nl.
Toch wordt er verwacht dat er een goede samenwerking tussen de IT'er en de orthopedagoog zal ontstaan en er een eerste productversie zal ontwikkeld worden tegen het einde van het onderzoek.
Een verwachting in de toekomst, na de bachelorproef, is dat een instantie de ontwikkeling van de applicatie verder in handen neemt. Er kan bijvoorbeeld contact opgenomen worden met het zorglab van Vives. Het zorglab verdiept en verbreedt de expertise over zorgtechnologie en interprofessioneel samenwerken en vertaalt deze kennis via kennisvalorisatie naar eindgebruikers, zorgverleners, bedrijven en het onderwijs.~\autocite{Vives}



%%---------- Andere bijlagen --------------------------------------------------
% TODO: Voeg hier eventuele andere bijlagen toe
\chapter{Stappenplan proefafname}
\label{appendix:stappenplan proefafname}

De verschillende stappen die de onderzoeker heeft gevolgd tijdens elke afname van de proef zijn genoteerd en afgeprint. Hierdoor kon de onderzoeker deze steeds bij zich houden om zeker niet af te wijken van het vaste plan. Het stappenplan is opgenomen in deze bijlage.

\section{Deel 1}
\begin{itemize}
    \item Vraag aan de deelnemer om de introductietekst te lezen over stemgestuurde assistenten.
    \item Vraag aan de deelnemer om de algemene vragen op de eerste pagina van het formulier in te vullen.
    \item Vijns dat de deelnemer drie Engelse commando’s zal geven aan een stemgestuurde assistent. Zeg dat de assistent klaar staat op de laptop en dat je terwijl het commando ook opneemt. Plaats de laptop rechtover de deelnemer en vertel hem dat hij mag voorover buigen terwijl hij het commando geeft.
\end{itemize}

\section{Deel 2}
\begin{itemize}
    \item Vraag aan de deelnemer om het eerste Engelse commando te lezen, maar benadruk dat hij het nog NIET mag uitspreken.
    \item Neem het eerste commando van de deelnemer op, terwijl hij zogezegd tegen de assistent praat.
    \item Vraag aan de deelnemer om het tweede Engelse commando te lezen, maar benadruk dat hij het nog NIET mag uitspreken.
    \item Neem het tweede commando van de deelnemer op, terwijl hij zogezegd tegen de assistent praat.
    \item Vraag aan de deelnemer om het derde Engelse commando te lezen, maar benadruk dat hij het nog NIET mag uitspreken.
    \item Neem het derde commando van de deelnemer op, terwijl hij zogezegd tegen de assistent praat.
\end{itemize}

\section{Deel 3}
\begin{itemize}
    \item Vertel dat het vreemd is dat hij geen enkel commando heeft gehoord, dus dat we het daarom eens in het Nederlands gaan proberen.
    \item Vraag aan de deelnemer om het eerste Nederlandse commando te lezen, maar benadruk dat hij het nog NIET mag uitspreken.
    \item Neem het eerste commando van de deelnemer op, terwijl hij zogezegd tegen de assistent praat.
    \item Vraag aan de deelnemer om het tweede Nederlandse commando te lezen, maar benadruk dat hij het nog NIET mag uitspreken.
    \item Neem het tweede commando van de deelnemer op, terwijl hij zogezegd tegen de assistent praat.
    \item Vraag aan de deelnemer om het derde Nederlandse commando te lezen, maar benadruk dat hij het nog NIET mag uitspreken.
    \item Neem het derde commando van de deelnemer op, terwijl hij zogezegd tegen de assistent praat.
\end{itemize}

\section{Deel 4}
\begin{itemize}
    \item Verduidelijk aan de deelnemer dat hij niet tegen een assistent heeft gepraat, maar dat dit nu wel zal gebeuren.
    \item Vraag aan de deelnemer de vragen met beschrijving te lezen die hij zal moeten beantwoorden. Verduidelijk dat het gaat over de kwaliteit van de uitspraak van de assistent en de manier waarop hij praat en niet over de inhoud van het antwoord. Het antwoord zal ook op het scherm komen, maar raad aan om vooral heel goed te luisteren.
\end{itemize}

\section{Deel 5}
\begin{itemize}
    \item Vraag aan de deelnemer de drie commando’s te geven aan Alexa en goed te luisteren naar de antwoorden.
    \item Vraag aan de deelnemer de vragen over Alexa in te vullen.
    \item Vraag aan de deelnemer de drie commando’s te geven aan Google Assistant en goed te luisteren naar de antwoorden.
    \item Vraag aan de deelnemer de vragen over Google Assistant in te vullen.
    \item Vraag aan de deelnemer de drie commando’s te geven aan Google Assistant NL en goed te luisteren naar de antwoorden.
    \item Vraag aan de deelnemer de vragen over Google Assistant NL in te vullen.
\end{itemize}

\section{Deel 6}
\begin{itemize}
    \item Vraag aan de deelnemer de algemene vragen op de laatste pagina in te vullen.
\end{itemize}

\chapter{Stappenplan analyse}
\label{appendix:stappenplan analyse}

Om de huidige resultaten te verkrijgen zijn er instructies uitgevoerd in R, gebruikmakend van R Studio. Wanneer men de analyse opnieuw wilt uitvoeren, kunnen de stappen, opgenomen in deze bijlage, gevolgd worden. Om het overzichtelijk te houden wordt er af en toe verwezen naar externe scripts. Deze zijn te vinden onder de map onderzoek/scripts in de repository beschreven in \ref{s:verwijzing naar repository}.

\section{De spraaksynthese van de assistenten}
\subsection{Inlezen van csv-file}
\begin{lstlisting}
results <- read.csv("[pad naar map]/spraakkwaliteit stemgestuurde assistenten.csv", sep=",")
\end{lstlisting}

\subsection{Installeren van de nodige packages}
\begin{lstlisting}
Install.packages(“eeptools”)
Install.packages(“reshape2”)
Install.packages(“stringr”)
install.packages(“formattable”)
\end{lstlisting}

\subsection{Voorbereiden van de data}
\begin{lstlisting}
Run script dataformat_leeftijden.R
Run script dataformat_vergelijking_assistenten_boxplots.R
Run script dataformat_vergelijking_assistenten_barplots.R
Run script dataformat_per_eigenschap.R
Run script dataformat_per_assistent.R
Run script dataformat_ttesten.R
\end{lstlisting}

\subsection{Leeftijd van de deelnemers}
\begin{lstlisting}
boxplot(leeftijd, main="Leeftijd van de deelnemers", yaxt='n')
axis(side=2, at=seq(0, 100, by = 5))
\end{lstlisting}

\subsection{Vergelijking van de assistenten per eigenschap}
\subsubsection{Boxplot van alle scores t.o.v. de assistenten}
\begin{lstlisting}
boxplot(coreResultsLong$score~coreResultsLong$assistant, main='Alle gegeven scores op de spraakkwaliteit van de assistenten', xlab="assistenten", ylab = 'score op vijf')
\end{lstlisting}

\subsubsection{Boxplots van de scores per eigenschap van de assistenten}
\paragraph{Verstaanbaarheid}
\begin{lstlisting}
boxplot(verstaanbaarheid$score~verstaanbaarheid$assistant, main="Gegeven scores op de verstaanbaarheid van de assistenten", xlab="assistent", ylab = "score")
\end{lstlisting}

\paragraph{Menselijkheid}
\begin{lstlisting}
boxplot(menselijkheid$score~menselijkheid$assistant, main="Gegeven scores op de menselijkheid van de assistenten", xlab="assistent", ylab = "score")
\end{lstlisting}

\paragraph{Levendigheid}
\begin{lstlisting}
boxplot(levendigheid$score~levendigheid$assistant, main="Gegeven scores op de levendigheid van de assistenten", xlab="assistent", ylab = "score")
\end{lstlisting}

\paragraph{Tempo}
\begin{lstlisting}
boxplot(tempo$score~tempo$assistant, main="Gegeven scores op het tempo van de assistenten", xlab="assistent", ylab = "score")
\end{lstlisting}

\paragraph{Emotionaliteit}
\begin{lstlisting}
boxplot(gevoel$score~gevoel$assistant, main="Gegeven scores op de aanwezigheid van gevoel bij de assistenten", xlab="assistent", ylab = "score")
\end{lstlisting}

\subsubsection{Barplots van de scores per eigenschap van de assistenten}
\paragraph{Verstaanbaarheid}
\begin{lstlisting}
barplot <- barplot(bpVerstaanbaarheid, main="verstaanbaarheid", xlab="score", col=c("lightblue","red", "yellow"), legend = rownames(bpVerstaanbaarheid), beside=TRUE)
text(x = barplot, y=bpVerstaanbaarheid, label = bpVerstaanbaarheid, pos = 3, cex = 0.8)
\end{lstlisting}

\paragraph{Menselijkheid}
\begin{lstlisting}
barplot <- barplot(bpMenselijkheid, main="menselijkheid", xlab="score", col=c("lightblue","red", "yellow"), legend = rownames(bpMenselijkheid), beside=TRUE)
text(x = barplot, y=bpMenselijkheid, label = bpMenselijkheid, pos = 3, cex = 0.8)
\end{lstlisting}

\paragraph{Levendigheid}
\begin{lstlisting}
barplot <- barplot(bpLevendigheid, main="levendigheid", xlab="score", col=c("lightblue","red", "yellow"), legend = rownames(bpLevendigheid), beside=TRUE)
text(x = barplot, y=bpLevendigheid, label = bpLevendigheid, pos = 3, cex = 0.8)
\end{lstlisting}

\paragraph{Tempo}
\begin{lstlisting}
barplot <- barplot(bpTempo, main="tempo", xlab="score", col=c("lightblue","red", "yellow"), legend = rownames(bpTempo), beside=TRUE)
text(x = barplot, y=bpTempo, label = bpTempo, pos = 3, cex = 0.8)
\end{lstlisting}

\paragraph{Emotionaliteit}
\begin{lstlisting}
barplot <- barplot(bpEmotionaliteit, main="emotionaliteit", xlab="score", col=c("lightblue","red", "yellow"), legend = rownames(bpEmotionaliteit), beside=TRUE)
text(x = barplot, y=bpEmotionaliteit, label = bpEmotionaliteit, pos = 3, cex = 0.8)
\end{lstlisting}

\subsection{Vergelijking van de eigenschappen per assistent}
\subsubsection{Boxplots van de scores per eigenschap voor één assistent}
\paragraph{Alexa}
\begin{lstlisting}
boxplot(alexa$score~alexa$eigenschap, main="Gegeven scores op de eigenschappen van Alexa", xlab="eigenschap", ylab = "score")
\end{lstlisting}

\paragraph{Google Assistant}
\begin{lstlisting}
boxplot(ga$score~ga$eigenschap, main="Gegeven scores op de eigenschappen van Google Assistant", xlab="eigenschap", ylab = "score")
\end{lstlisting}

\paragraph{Google Assistant NL}
\begin{lstlisting}
boxplot(ganl$score~ganl$eigenschap, main="Gegeven scores op de eigenschappen van Google Assistant in het Nederlands", xlab="eigenschap", ylab = "score")
\end{lstlisting}

\subsection{De gemiddelde score en standaardafwijking van alle eigenschappen per assistent}
\begin{lstlisting}
attach(coreResultsLong)
mean_sd_scores <- setNames(aggregate(x = score, by=list(eigenschap, assistant), FUN = function(x) c(mean = mean(x), sd = sd(x))),c("eigenschap", "assistant", ""))
mean_sd_scores
mean_sd_scores_ordered <- mean_sd_scores[order(-mean_sd_scores$mean),]
library(formattable)
formattable(mean_sd_scores_ordered)
\end{lstlisting}

\subsubsection{t.testen}
\begin{lstlisting}
t.test(emotionaliteit_scores_Alexa,emotionaliteit_scores_GANL,alternative = "greater", paired = TRUE)
t.test(emotionaliteit_scores_GA,ttest_scores_GANL,alternative = "greater", paired = TRUE)
t.test(levendigheid_scores_Alexa,levendigheid_scores_GANL,alternative = "greater", paired = TRUE)
t.test(levendigheid_scores_GA,levendigheid_scores_GANL,alternative = "greater", paired = TRUE)
t.test(menselijkheid_scores_Alexa,menselijkheid_scores_GANL,alternative = "greater", paired = TRUE)
t.test(menselijkheid_scores_GA,menselijkheid_scores_GANL,alternative = "greater", paired = TRUE)
t.test(tempo_scores_Alexa,tempo_scores_GANL,alternative = "greater", paired = TRUE)
t.test(tempo_scores_GA,tempo_scores_Alexa,alternative = "greater", paired = TRUE)
t.test(verstaanbaarheid_scores_GA,verstaanbaarheid_scores_GANL,alternative = "greater", paired = TRUE)
\end{lstlisting}

\section{De spraakherkenning van de assistenten}
\subsection{Inlezen van csv-file}
\begin{lstlisting}
results <- read.csv("[pad naar map]/spraakkwaliteit stemgestuurde assistenten.csv", sep=",")
\end{lstlisting}
\subsection{Voorbereiden van de data}
\begin{lstlisting}
Run script dataformat_tabel_alle_teksten.R
\end{lstlisting}

\subsection{Een overzicht van de gevormde tekst}
\begin{lstlisting}
library(formattable)
formattable(texttable)
\end{lstlisting}


%%---------- Referentielijst --------------------------------------------------
\printbibliography[heading=bibintoc]
%\addcontentsline{toc}{chapter}{\textcolor{maincolor}{\IfLanguageName{dutch}{Bibliografie}{Bibliography}}}

\end{document}
