%%=============================================================================
%% Inleiding
%%=============================================================================

\chapter{Inleiding}
\label{ch:inleiding}
Sinds het bestaan van de computer zijn we het gewoon om externe apparatuur zoals een toetsenbord of muis te hanteren om ermee te communiceren. Dit zorgde doorheen de jaren voor heel wat frustratie en was het voor velen bijna een verplichting om hiermee te leren werken. Toch gebruiken we van mens tot mens een ander soort communicatiemiddel, een manier die ons van nature wordt aangeleerd, de gesproken conversatie.
Een instinctieve kunst die mensen al duizenden jaren onder de knie hebben. Het is één van de eerste en belangrijkste vaardigheden die een kind leert. Een kunst, eigen aan de mens, die nooit hoofdzakelijk is gebruikt als communicatiemiddel met een computer. Toch werd hier al tientallen jaren geleden met geëxperimenteerd. Vandaag de dag komt de verwachting om op dezelfde manier te communiceren met de computer dan als we doen met mensen dichtbij. De laatste jaren is de kwaliteit van spraaksystemen gegroeid door onder meer de doorbraak in neurale netwerken. De spraakkwaliteit heeft zo een niveau bereikt dat de conversatie tussen mens en computer natuurlijk begint aan te voelen.

Het is de reden waarom Apple in 2011 een eerste persoonlijke spraakassistent lanceerde, genaamd Siri. Andere bedrijven volgden snel. Google ontwikkelde de Google Assistant Microsoft bouwde Cortana en Amazon, het bedrijf met één van de grootste webshobs ter wereld, kwam met Alexa op de proppen. De spraakassistenten kunnen tegenwoordig niet alleen gebruikt worden via de smartphone en laptop, maar ook via de Smart Speaker, een slimme luidspreker met ingebouwde spraakassistent.

Beeld het je in. Uw persoonlijke assistent heeft je net gewekt. Wanneer je opstaat, vertelt hij je dat het vandaag 15 graden wordt en overwegend bewolkt. Wanneer je beneden komt overloopt hij de eerste nieuwsfeiten van de dag. Je vindt jouw kamerjas niet en vraagt dan maar aan de assistent om de temperatuur 5 graden te verhogen. Tijdens het ontbijt krijg je nog te horen hoe de huidige verkeerssituatie is naar het werk. Door de lange file ga je pas om 09.07u toekomen op kantoor. Dringend tijd dus om te vertrekken. Assistent, doe jij even de deur voor me op slot?

Je kunt er je wel iets bij voorstellen. Het klinkt allemaal sciencefiction, maar het feit is dat dit al dagelijkse realiteit is. De slimme luidsprekers doen het goed. In de Verenigde Staten hebben één op de vijf volwassenen al toegang tot een smart speaker met stemassistent.~\autocite{Passies2018} De opkomst in België blijft nog even uit, maar lijkt binnenkort te arriveren met de Belgische variant van de Google Assistant. Ook Google Home, de slimme luidspreker van Google zou binnenkort op de Belgische markt verschijnen. In Nederland is het apparaat al langer beschikbaar, namelijk sinds eind oktober 2018.~\autocite{Haenen2018}

Omdat de slimme luidspreker nog geen grote vlucht heeft genomen in België zijn het aantal bestaande applicaties voor deze technologie nog schaars. Mijn onderzoek begint met het verhaal over mijn zoektocht naar een gepast doel voor de spraakassistent. Een doel met niet alleen interessante mogelijkheden, maar ook een doel voor een groep die geschikt is om met de huidige middelen onderzocht te worden.

Op 2 april 2019 kondigde het Rode Kruis aan dat het een mobiele applicatie had ontwikkeld om te helpen met het verlenen van eerste hulp bij ongevallen. Ik zag hier direct mijn kans in om naast de mobiele applicatie een toepassing voor de spraakassistent te voorzien.

Het eerste en grootste deel van het onderzoek probeert een antwoord te vinden op volgende onderzoeksvraag: Welke spraakassistent is het meest geschikt om te helpen bij het verlenen van eerste hulp bij ongevallen? Daarna worden de eerste stappen genomen in het ontwikkelen van een EHBO app voor de meest geschikte spraakassistent die kan gebruikt worden door de Vlaming. De Vlaming en niet de Belg, omdat de applicatie om te beginnen in maar één taal zal beschikbaar zijn, het Nederlands.

Spraakherkenning en spraaksynthese zijn twee belangrijke vereisten voor de werking van een spraakassistent. De kwaliteit van deze functies zijn ook belangrijk voor het geven van instructies bij EHBO. Er werd een onderzoek gevoerd naar de kwaliteit van de spraakherkenning en spraaksynthese van drie assistenten. Het gaat om de Nederlandstalige en Engelstalige Google Assistant en de Engelstalige Alexa.

Dit onderzoek probeert verschillen aan te tonen in kwaliteit van spraakherkenning en spraaksynthese tussen de drie assistenten. Dertig participanten stelden drie EHBO-gerelateerde vragen aan elke assistent. Er werd geprobeerd om te meten hoe correct de vragen werden omgevormd van spraak naar tekst. De participanten beoordeelden ook de antwoorden van de assistenten op enkele eigenschappen van spraakkwaliteit.

De rest van deze bachelorproef is als volgt opgebouwd:

In Hoofdstuk \ref{ch:zoektocht} wordt het verhaal gebracht van mijn zoektocht naar een gepast doel voor de applicatie.

In Hoofdstuk~\ref{ch:stand-van-zaken} wordt een overzicht gegeven van de stand van zaken binnen het onderzoeksdomein, op basis van een literatuurstudie.

In Hoofdstuk~\ref{ch:methodologie} wordt de methodologie toegelicht en worden de gebruikte onderzoekstechnieken besproken om een antwoord te kunnen formuleren op de onderzoeksvragen.

In Hoofdstuk~\ref{ch:resultaten} worden de resultaten getoond uit het onderzoek die de drie assistenten vergelijkt.

In Hoofdstuk~\ref{ch:eerstehulpapp} wordt een kort beeld geschetst van hoe de eerste versie van de EHBO-applicatie er zal uitzien.

In Hoofdstuk~\ref{ch:conclusie}, tenslotte, wordt de conclusie gegeven en een antwoord geformuleerd op de onderzoeksvragen. Daarbij wordt ook een aanzet gegeven voor toekomstig onderzoek binnen dit domein.

\section{De repository}
\label{s:verwijzing naar repository}
Alle extra documenten kunnen gevonden worden in de \href{https://github.com/JorgeReyniers/bachelorproef-spraakassistent-in-gesprekscontext}{repository} van de bachelorproef op Github.
De link naar de repository is \textbf{https://github.com/JorgeReyniers/bachelorproef-spraakassistent-in-gesprekscontext}.