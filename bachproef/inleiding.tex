%%=============================================================================
%% Inleiding
%%=============================================================================

\chapter{Inleiding}
\label{ch:inleiding}
Sinds het bestaan van de computer zijn we het gewoon om externe apparatuur zoals een toetsenbord of muis te hanteren om ermee te communiceren. Dit zorgde doorheen de jaren voor heel wat frustratie en was het bijna een verplichting voor velen om hiermee te leren werken. Toch gebruiken we van mens tot mens een ander soort communicatiemiddel, een manier die ons van nature wordt aangeleerd, de gesproken conversatie.
Een instinctieve kunst die mensen al duizenden jaren onder de knie hebben. Het is één van de eerste en belangrijkste vaardigheden die een kind leert. Een kunst, eigen aan de mens, die nooit hoofdzakelijk is gebruikt als communicatiemiddel met een computer. Toch werd hier al tientallen jaren geleden met geëxperimenteerd. Vandaag de dag komt de verwachting om op dezelfde manier te communiceren met de computer dan als we doen met mensen dichtbij.

Op 2 april 2019 kwam het Rode Kruis met het nieuws dat ze een mobiele applicatie hebben ontwikkeld voor het geven van eerste hulp bij ongevallen. Dit nieuws was dan ook de aanleiding om mijn onderzoek naar verschillende spraakgestuurde technologieën te koppelen aan het geven van EHBO. 

\section{Probleemstelling}
\label{sec:probleemstelling}

\section{Onderzoeksvraag}
\label{sec:onderzoeksvraag} 

\section{Onderzoeksdoelstelling}
\label{sec:onderzoeksdoelstelling}

\section{Opzet van deze bachelorproef}
\label{sec:opzet-bachelorproef}


